\subsection{Desktops vs. Laptops}

If you are using one of our provided desktops, all installation procedures have been carried out. You merely need to go to the \verb+lxmls-toolkit-student+ folder inside your home directory and start working! You may go directly to section \ref{sec:Python}.

If you wish to use your own laptop, you will need to install Python, all required Python libraries, and the LXMLS code base. Please follow the instructions appropriate for your operating system.

\subsection{Installing in Windows or Mac}

While Python is easy to install, some of the Python modules are based in C and thus require a compiler to be installed from scratch. Fortunately, there are some companies which distribute Windows and Mac versions of Python including all C-based libraries which we need, for free. We recommend one of these:

\begin{itemize}
\item Anaconda 
\begin{verbatim}
http://continuum.io/downloads
\end{verbatim}
%
\item Canopy
\begin{verbatim}
https://store.enthought.com/downloads/
\end{verbatim}
%
\item WinPython (Windows only)
\begin{verbatim}
http://sourceforge.net/projects/winpython/files/
\end{verbatim}
%
\item Using the \verb+port+ command (Mac only); please follow the instructions at
\begin{verbatim}
http://www.scipy.org/install.html
\end{verbatim}
%
\end{itemize}

Just go to the corresponding websites and follow the instructions for installation. Make sure you install a 2.7.x version, not a 3.x version!

%Any python 2.7.x version will do. You can check the version by  
%
%If you do not have python, you can install the latest 2.7+ python from the official website at
%
%\begin{verbatim}
%https://www.python.org/downloads/ 
%\end{verbatim}
%
%Do not install python 3.x.x versions, this is a backwards incompatible version that can not use most of the tools of the labs. 
%
%Then we will need to instal some specific Python modules and the ipython shell. The easiest way to do this is by installing a python module manager.
%
%
%
%
%Now we need the ipython interactive shell. Again we use the command line (be sure to exit python first by using exit() or ctrl+d).
%
%\begin{verbatim}
%>ipython
%\end{verbatim}
%
%if this is recognized then great
% 
%
%If you do not know if you have a module manager installed open a command window
%
%\begin{verbatim}
%press the windows-key + R 
%write cmd and press enter  
%try the command pip 
%try the command easy_install 
%\end{verbatim}
%


%After hovering the mouse over downloads, you will see various options for installation. You can instal the version 2.7.x, any number for x is ok. Please do not instal version 3 or superior. If you already have a python 2.7 (any version) installed, you dont need to install another one, it will work. If you have older versions of python it might still work. In any case please uninstall any previous python version before installing a new one. This is a frequent source of problems on all systems.
%
%Now for the different python modules needed. The most straightforward way to instal the modules is through a python package manager, we recommend pip 
%
%[lot to do here ...]
%
%\begin{verbatim}
%pip install -r pip-requirements.txt 
%\end{verbatim}
%

\subsection{Installing in Linux}

Installation of Python and its related packages in Linux is best done from your distribution's package manager. Please follow the instructions for your distribution, available at
\begin{verbatim}
http://www.scipy.org/install.html
\end{verbatim}

If you are presented with a choice, please install a Python 2.7.x version, not a 3.x version!

\subsection{Testing your installation}

To check whether your Python installation is working, start by invoking the Python shell. You can do that by running the command \verb+python+ in your Terminal in Linux or Mac, or the file \verb+python.exe+ from the folder where Python was installed in, for Windows.

In the Python interpreter, the very first line should read like
\begin{verbatim}
Python 2.7.3 (default, Feb 27 2014, 19:58:35)
\end{verbatim}
%
In this case, the Python version was 2.7.3. Please make sure that in your case this reads \verb+2.7.x+, where \verb+x+ is some number. Newer versions such as 3.4 have not been tested with our code.

To test whether Numpy is working, type the following two lines in the Python interpreter:
%
\begin{verbatim}
import numpy
numpy.test()
\end{verbatim}
%
The last two lines should read something like
%
\begin{verbatim}
OK (KNOWNFAIL=3, SKIP=4)
<nose.result.TextTestResult run=3161 errors=0 failures=0>
\end{verbatim}

Finally, to check that all the modules we will use work, please run the \verb+all_imports.py+ file found in the \verb+lxmls-toolkit-student+ folder.

\subsection{Downloading the labs version from GitHub}

The code of LxMLS is available online at GitHub. There are two branches of the code: the \verb+master+ branch contains fully functional code. The \verb+student+ branch contains the same code with some parts deleted, which you must complete in the following exercises. To download the \verb+student+ code, go to

\begin{verbatim}
https://github.com/gracaninja/lxmls-toolkit
\end{verbatim}

\noindent and select the \verb+student+ branch in the dropdown menu. This will reload the page to the corresponding branch. Now you just need to click the \verb+download ZIP+ button to obtain the lab tools in a zip format:

\begin{verbatim}
lxmls-toolkit-student.zip
\end{verbatim}

After this you can unzip the file where you want to work and enter this folder.

\subsection{Working on the Exercises}

Please note that in order for the exercises to work you need to launch your scripts or use interpreters from inside the  \textit{lxmls-toolkit-student} folder.

One detail more and you are off to go. In the student branch we provide the \verb+solve.py+ script. This can be used to solve the exercises of each day, \emph{e.g.}

\begin{verbatim}
python ./solve.py day1
\end{verbatim}

\noindent You can also undo the solving of an exercise by using

\begin{verbatim}
python ./solve.py --undo day1
\end{verbatim}

Note that this script just downloads the master or student versions of certain files from the GitHub repository. It needs an Internet connection. Since some exercises require you to have the exercises of the previous days completed, the monitors may ask you to use this function. Remember to save your own version of the code, otherwise it will be overwritten!

