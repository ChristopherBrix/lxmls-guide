\subsection{Desktops vs. Laptops}

If you have decided to use one of our provided desktops, all installation
procedures have been carried out. You merely need to go to the
\verb+lxmls-toolkit-student+ folder inside your home directory and start
working! You may go directly to section \ref{sec:SolvingExercises}. If you
wish to use your own laptop, you will need to install Python, the 
required Python libraries, and the LXMLS code base. It is important that you
do this as soon as possible (preferably before the school starts) to avoid
unnecessary delays. Please follow the install instructions. 

\subsection{Downloading the labs version from GitHub}

The code of LxMLS is available online at GitHub. There are two branches of the
code: the \verb+master+ branch contains fully functional code. The
\verb+student+ branch contains the same code with some parts deleted, which you
must complete in the following exercises. To download the \verb+student+ code,
go to

\begin{verbatim}
https://github.com/LxMLS/lxmls-toolkit
\end{verbatim}

\noindent and select the \verb+student+ branch in the dropdown menu. This will
reload the page to the corresponding branch. Now you just need to click the
\verb+download ZIP+ button to obtain the lab tools in a zip format:

\begin{verbatim}
lxmls-toolkit-student.zip
\end{verbatim}

After this you can unzip the file where you want to work and enter the unzipped
folder. This will be the place where you will work.

\subsection{Installing The Tools from Scratch}

If you are new to Python the best option right now is the Anaconda suite. You
can find installers for Windows, Linux and OSX platforms here

\begin{verbatim}
https://www.continuum.io/downloads
\end{verbatim}

\textbf{Important}: Be sure to install the version for Python 2.7. This
installer will already provide most of the packages you need to work with the
school toolkit. The only package missing will be Theano, you may follow the
instructions here to install it

\begin{verbatim}
http://deeplearning.net/software/theano/install.html#anaconda
\end{verbatim}

\noindent Note that you just need a normal Theano installation. You do not 
need the bleeding edge version and it is not recommended to install it.

\subsection{Installing for users with some Python experience}

If you are familiar with Python you will probably be used to the pip package
installer. In this case it might be more easy for you to install the packages
yourself using pip

\begin{verbatim}
pip install -r pip-requirements.txt 
\end{verbatim}

You may need to pre-append sudo to the system call depending on the platform.

\noindent this will install the following packages

\begin{verbatim}
jupyter
numpy
scipy
matplotlib
pyyaml
nltk
theano 
\end{verbatim}

To check that all the modules we will use work, please run the
\verb+all_imports.py+ file found in the \\ \verb+lxmls-toolkit-student+ folder.


\subsection{Virtual Environments}

If you are an experienced Python user and you do not want to add many packages
to your existing installation, you can use virtualenv.

\begin{verbatim}
sudo pip install virtualenv
virtualenv venv 
\end{verbatim}

\noindent This will create a local installation of Python on your work folder. To
activate it use 

\begin{verbatim}
source venv/bin/activate
\end{verbatim}

\noindent then you can install packages locally with no need for root privileges and without touching your original installation.

\begin{verbatim}
pip install -r pip-requirements.txt
\end{verbatim}

\noindent To exit the environment use 

\begin{verbatim}
deactivate
\end{verbatim}

\noindent If you want to make use of the modules already installed in your main Python installation use

\begin{verbatim}
virtualenv venv --system-site-packages
\end{verbatim}

instead of the conventional call. Bear in mind that using virtual environments can create some problems. For once, the Python binary will be in another location. You will therefore have to be careful when using e.g. shebangs to specify it. The Jupyter-notebook is also know to give problems unless you also install it with the virtual environment. For these reasons, virtual environments are only recommended for students that feel comfortable with Python. 

\subsection{Deciding on the IDE and interactive shell to use}

An Integrated Development Environments (IDE) includes a text editor and various tools to debug and interpret complex code. \textbf{Important:} As the labs progress you will need an IDE, or at least a good editor and knowledge of pdb/ipdb. This will not be obvious the first days since we will be seeing simpler examples.

As IDE we recommend PyCharm, but feel free to use the software you feel more comfortable with. PyCharm and other well known IDEs like Spyder are provided with the conda installation.

Aside of an IDE, you will need an interactive command line to run commands. This is very useful to explore variables and functions and quickly debug the exercises. For the most complex exercises you will still need an IDE to modify particular segments of the provided code. As interactive command line we recommend the Jupyter notebook. This also comes installed with anaconda and is part of the pip-installed packages. The Jupyter notebook is described in the next section. In case you run into problems or you feel uncomfortable with the Jupyter notebook you can use the simple iPython command line.

\subsection{Jupyter Notebook}

Jupyter is also a good choice when writing Python code. It is an interactive computational environment for data science and scientific computing, where you can combine code execution, rich text, mathematics, plots and rich media. The Jupyter Notebook is a web application that allows you to create and share documents, which contains live code, equations, visualizations and explanatory text. It is very popular in the areas of data cleaning and transformation, numerical simulation, statistical modeling, machine learning and so on. It supports more than 40 programming languages, including all those popular ones used in Data Science such as Python, R, and Scala. It can also produce many different types of output such as images, videos, LaTex and JavaScript. More over with its interactive widgets, you can manipulate and visualize data in real time.

\noindent The main features and advantages using the Jupyter Notebook are the
following:

\begin{itemize}

\item In-browser editing for code, with automatic syntax highlighting, indentation, and tab completion/introspection.

\item The ability to execute code from the browser, with the results of computations attached to the code which generated them.

\item Displaying the result of computation using rich media representations, such as HTML, LaTeX, PNG, SVG, etc. For example, publication-quality figures rendered by the matplotlib library, can be included inline.

\item In-browser editing for rich text using the Markdown markup language, which can provide commentary for the code, is not limited to plain text.

\item The ability to easily include mathematical notation within markdown cells using LaTeX, and rendered natively by MathJax.

\end{itemize}

\noindent A more detailed user guide can be found here:

\begin{verbatim}
http://jupyter-notebook-beginner-guide.readthedocs.io/en/latest/index.html
\end{verbatim}
