\subsection{Running Python}

You can access and run Python interactively, simply by running the
\texttt{python} command. Alternatively, you can save your program to a file
and run python on it:

\begin{verbatim}
python yourfile.py
\end{verbatim}

In these lab sessions, we are going to be using Python in interactive mode
several times. The standard Python interface is not very friendly, though.
IPython, which stands for \emph{interactive Python}, is an improved Python
shell. It saves your command history between sessions, has basic auto-complete,
and has internal support for interacting with graphs through matplotlib.

IPython is also designed to facilitate running parallel code of clusters of
machines, but we will not make use of that functionality.

To run IPython, simply run \texttt{IPython} on your command line. For
interactive numeric use, the \texttt{-pylab} flag imports numpy and matplotlib
for you and sets up interactive graphs:

\begin{verbatim}
IPython -pylab
\end{verbatim}

In some systems you may need to run the command lower-cased:

\begin{verbatim}
ipython -pylab
\end{verbatim}

\subsubsection{Help and Documentation}

There are several ways to get help on IPython:

\begin{itemize}
\item Adding a question mark to the end of a function or variable and pressing Enter brings up associated documentation. Unfortunately, not all packages are well documented. Numpy and matplotlib (the two libraries we will extensively use in the lab sessions) are pleasant exceptions;
\item \code{help('print')} gets the online documentation for the \code{print} keyword;
\item \code{help()}, enters the help system.
\end{itemize}

When at the help system, type \code{q} to exit.

For more information on IPython~\citep{PER-GRA:2007}, check the website: \url{http://ipython.scipy.org/moin/}


\subsubsection{Profiling}


If you are interested in checking the performance of your program, you can use the
command \%prun in IPython (this is an IPython-only feature). For example:


\begin{python}
def myfunction(x):
    ...

%prun myfunction(22)
\end{python}

\subsubsection{Exiting}

Exit IPython by typing \code{exit()} or \code{quit()} (or typing CTRL-D).

\subsection{Python by Example}

%%%% Hello World

The first program of every programmer in every new language prints "Hello,
World!". In Python, this simply reads:

\begin{python}
 In[]: print "Hello, World!"
Out[]: Hello, World!
\end{python}


\subsubsection{Data Structures}

In Python, you can create lists of items with the following syntax:

\begin{python}
countries = ['Portugal','Spain','United Kingdom']
\end{python}

A string should be surrounded with apostrophes ('). You can access a list with
the following:

\begin{itemize}
 \item \code{len(L)}, which returns the number of items in L;
 \item \code{L[i]}, which returns the item at index i (the first item has index 0);
 \item \code{L[i:j]}, which returns a new list, containing the items between i and j. 
\end{itemize}

\begin{exercise}
 Use L[i:j] to return the countries in the Iberian Peninsula.
\end{exercise}

\subsubsection{Loops}

A loop allows a certain section of code to be repeated a certain number of
times. The loops continue until a stop condition is reached. For instance, when
a variable has reached a certain value or when the list you are iterating has
reached its end. In Python you have \code{while} and \code{for} loops.

The following two programs output exactly the same: the even numbers from 2 to
8.


\begin{python}
i = 2
while i < 10:
  print i  
  i += 2 
\end{python}

\begin{python}
for i in range(2,10,2):
    print i
\end{python}

The \code{range} function is built int to Python and it creates lists
containing arithmetic progressions. 

\begin{exercise}
David, John, Allysson and Anne are four of your colleagues in the Summer Course. Create a python program to greet all of them. The output should be\\
\begin{python} 
Hello, David!
Hello, John!
Hello, Allysson!
Hello, Anne!
\end{python}
Note that you have around 100 colleagues. You should use the data structures you have just learned to minimize the lines of code you are using in this exercise.
\end{exercise}


\subsubsection{Control Flow}

The \code{if} statement allows to control the flow of your program. The next
program makes a greeting that depends on the time of the day.

\begin{python}
if hour < 12:
    print 'Good morning!'
elif hour >= 12 and hour < 20:
    print 'Good afternoon!'
else:
    print 'Good evening!'
\end{python}

 
\subsubsection{Functions}

A function is a block of code that can be reused to perform a similar action. The following is a function in Python. 

\begin{python}
def greet(hour):
    if hour < 12:
        print 'Good morning!'
    elif hour >= 12 and hour < 20:
        print 'Good afternoon!'
    else:
        print 'Good evening!'
\end{python}

If you call the function \code{greet} with different hours of the day, the program will greet you accordingly.

\begin{exercise}
 Note that the previous code allows the hour to be less than 0 or more than 24. Change the code in order to indicate that the hour given as input is invalid. Your output should be something like:

\begin{python}
greet(50)
Invalid hour: it should be between 0 and 24.
greet(-5)
Invalid hour: it should be between 0 and 24.
\end{python}

\end{exercise}

\subsubsection{Indentation}

In Python, indentation is important. This is how Python differentiates between
nested and non-nested blocks of commands. For instance, consider the following
code and its output:

\begin{python}
a=1
while a <= 3:
    print a
    a += 1
\end{python}

\begin{python}
1
2
3
\end{python}


\begin{exercise}
Can you predict the output of the following code:

\begin{python}
a=1
while a <= 3:
    print a
a += 1
\end{python}

\end{exercise}

\subsection{Plotting in Python - Matplotlib}

Matplotlib is a plotting library for Python. It supports \textsc{2d} and
\textsc{3d} plots of various forms. It can show them interactively or save them
to a file (several output formats are supported).

\begin{python}
import numpy as np
import matplotlib.pyplot as plt

X = np.linspace(-4, 4, 1000)

plt.plot(X, X**2*np.cos(X**2))
plt.savefig("simple.pdf")
\end{python}


\begin{exercise}
Try running the following on IPython, which will introduce you to some of the basic numeric and plotting operations.

\begin{python}
# This will import the numpy library
# and give it the np abbreviation
import numpy as np

# This will import the plotting library
import matplotlib.pyplot as plt

# Linspace will return 1000 points,
# evenly spaced between -4 and +4
X = np.linspace(-4, 4, 1000)

# Y[i] = X[i]**2
Y = X**2

# Plot using a red line ('r')
plt.plot(X, Y, 'r')

# arange returns points ranging from -4 to +4
# (the upper argument is excluded!)
Ints = np.arange(-4,5)

# We plot these on top of the previous plot
# using blue circles (o means a little circle)
plt.plot(Ints, Ints**2, 'bo')

# You may notice that the plot is tight around the line
# Set the display limits to see better
plt.xlim(-4.5,4.5)
plt.ylim(-1,17)
\end{python}
\end{exercise}

%There are many options to \texttt{plt.plot} which manipulate the appearance of the plot.  
%You can see them all by querying the documentation using \texttt{plt.plot?} 

\subsection{Numpy}

Numpy is a library needed for scientific computing with Python. 

\subsubsection{Multidimensional Arrays}

The main object of numpy is the multidimensional array. A multidimensional
array is a table with all elements of the same type and can have several
dimensions.

Numpy provides various functions to access and manipulate multidimensional
arrays. In one dimensional arrays, you can index, slice, and iterate as you can
with lists. In a two dimensional array M, you can use perform these operations
along several dimensions.

\begin{itemize}
 \item M[i,j], to access the item in the ith row and j-th column; 
 \item M[i:j,:], to get the all the rows between the i-th and j-th;
 \item M[:,i], to get the i-th column of M.
\end{itemize}

Again, as it happened with the lists, the first item of every column and every row has index 0.

\begin{python}
import numpy as np
A = np.array([
    [1,2,3],
    [2,3,4],
    [4,5,6]])

A[0,:] # This is [1,2,3]
A[0] # This is [1,2,3] as well

A[:,0] # this is [1,2,4]

A[1:,0] # This is [ [2], [4] ]. Why?
        # Because it is the same as A[1:n,0] where n is the size of the array.
\end{python}

\subsubsection{Mathematical Operations}

There are many helpful functions in numpy. For basic mathematical operations, we have \code{np.log}, \code{np.exp},
\code{np.cos},\ldots with the expected meaning. These operate both on single
arguments and on arrays (where they will behave element wise).

\begin{python}
import matplotlib.pyplot as plt
import numpy as np

X = np.linspace(0, 4 * np.pi, 1000)
C = np.cos(X)
S = np.sin(X)

plt.plot(X, C)
plt.plot(X, S)
\end{python}

Other functions take a whole array and compute a single value from it. For
example, \code{np.sum}, \code{np.mean},\ldots These are available as both free
functions and as methods on arrays.

\begin{python}
import numpy as np

A = np.arange(100)
print np.mean(A)
print A.mean()

C = np.cos(A)
print C.ptp()
\end{python}

\begin{exercise}
Run the above example and lookup the \code{ptp} function/method (use the \texttt{?} functionality in ipython).
\end{exercise}


\begin{exercise}
Consider the following approximation to compute an integral

\[
\int_0^{1} f(x)dx \approx \sum_{i = 0}^{999} \frac{f(i/1000)}{1000}.
\]

Use numpy to implement this for $f(x) = x^2$. You should not need to use any
loops. The exact value is $1/3$. How close is the approximation?
\end{exercise}



% \subsection{Debugging}
% 
% There are a few options for debugging Python programs. Given that we are
% using IPython, we will explore its internal methods.
% 
% When you run a program in IPython and there is an unhandled error (uncaught
% exception), you can type \code{debug} to enter the debugger (alternatively, you
% could have ran the code inside the debugger).
% 
% Once inside the debugger, you can run debugging commands such as \code{step},
% \code{continue}, \code{up}/\code{down} (to move up and down the stack); or
% Python commands by prefixing them with \code{!}. The most common Python command
% you'll want to use is, of course, \code{print} to inspect the value of some
% variables and expressions.
% 
% 
% \begin{exercise}
% Use the debugger to debug the \texttt{buggy.py} script which attempts to
% repeatedly perform the following computation:
% 
% \begin{enumerate}
% \item Start $x_0 = 0$
% \item Iterate
% 
% \begin{enumerate}
% \item $x'_{t+1} = x_t + r$, where $r$ is a random variable.
% \item if $x'_{t+1} >= 1.$, then stop.
% \item if $x'_{t+1} <= 0.$, then $x_{t+1} = 0$
% \item else $x_{t+1} = x'_{t+1}$.
% \end{enumerate}
% \item Return the number of iterations.
% \end{enumerate}
% 
% This is attempting to predict the number of times that a ``drunk walker''
% (i.e., an agent who takes random steps to the left or to the right) takes to go
% from 0 to~1.
% 
% Having repeated this computation a number of times, the program prints the
% average. Unfortunately, the program has a few bugs, which you need to fix.
% \end{exercise}
% 
% 
