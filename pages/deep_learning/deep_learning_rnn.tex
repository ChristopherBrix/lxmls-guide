\section{Today's assignment}
Today's class will be focused on advanced deep learning concepts, mainly
Reccurrent Neural Networks (RNNs). As we will see RNNs can be handled as the
feed-forward computation graph seen in during the first day. 

\section{Recurrent Neural Networks: Backpropagation through Time}

\begin{exercise}
Convince yourself a RNN is just an MLP with inputs and outputs at various
layers. 

\textbf{TODO}:
\textbf{This should be a numpy example of backpropagation in a RNN akin to the MLP case}.
 
\end{exercise}

\section{The Scan operation in Theano}

\begin{exercise}
Scan is your friend, maybe. 

\textbf{TODO}:
\textbf{Simple examples of scan (both to see and implement)}
\end{exercise}

\section{RNNs}

\begin{exercise}
Complete the theano code for a RNN inside lxmls/deep\_learning/rnn.py. Use
exercise 6.1 for a numpy example and 6.2 to learn how to handle scan. 

Prepare the data
\begin{python}
# Load POS and compacify the indices
import lxmls.readers.pos_corpus as pcc
corpus    = pcc.PostagCorpus()
train_seq = corpus.read_sequence_list_conll("data/train-02-21.conll",
                                            max_sent_len=15, max_nr_sent=1000)
test_seq  = corpus.read_sequence_list_conll("data/test-23.conll",
                                            max_sent_len=15, max_nr_sent=1000)
dev_seq   = corpus.read_sequence_list_conll("data/dev-22.conll",
                                            max_sent_len=15, max_nr_sent=1000)
# Redo indices so that they are consecutive. Also cast all data to numpy arrays
# of int32 for compatibility with GPUs and theano.
train_seq, test_seq, dev_seq = pcc.compacify(train_seq, test_seq, dev_seq,
                                             theano=True)
\end{python}

After you have implemented the forward-pass correctly you should be able to run
this POS example. First define your model
\begin{python}
# CREATE RNN TO PREDICT POS TAGS 
import numpy as np
import theano
import theano.tensor as T
import lxmls.deep_learning.rnn as rnns

# DEFINE MODEL
# Extract word embeddings for the vocabulary used. Download embeddings if
# not available.
import os
if not os.path.isfile('data/senna_50'):
    rnns.download_embeddings('senna_50','data/senna_50')
E = rnns.extract_embeddings('data/senna_50', train_seq.x_dict)

# CONFIG 
n_words = E.shape[0]                        # Number of words
n_emb   = E.shape[1]                        # Size of word embeddings
n_hidd  = 20                                # Size of the recurrent layer
n_tags  = len(train_seq.y_dict.keys())      # Number of POS tags
# SYMBOLIC VARIABLES
_x      = T.ivector('x')                    # Input words indices
# Define the RNN
rnn     = rnns.RNN(E, n_hidd, n_tags)
# Forward
_p_y    = rnn._forward(_x)

\end{python}

Now define the training configuration

\begin{python}

# DEFINE TRAINING 
# CONFIG
lrate   = 0.5  # Learning rate          
n_iter  = 20   # Number of iterations
# SYMBOLIC VARIABLES
_y      = T.ivector('y')                   # True output tags indices
# Train cost
_F      = -T.mean(T.log(_p_y)[T.arange(_y.shape[0]), _y]) 
# Total prediction error 
_err    = T.sum(T.neq(T.argmax(_p_y,1), _y))

# SGD UPDATE RULE
updates = [(_par, _par - lrate*T.grad(_F, _par)) for _par in rnn.param] 

# COMPILE ERROR FUNCTION, BATCH UPDATE
err_sum      = theano.function([_x, _y], _err)
batch_update = theano.function([_x, _y], _F, updates=updates)

\end{python}

Now you can run training 

\textbf{TODO}: Experience about pre-trained embeddings   

\begin{python}

# TRAIN MODEL WITH SGD
#TODO: Merge this code with lxmls/deep_learning/sgd.py
# Function computing accuracy for a sequence of sentences
def accuracy(seq):
    err = 0
    N   = 0
    for n, seq in enumerate(seq):
        err += err_sum(seq.x, seq.y)
        N   += seq.y.shape[0]
    return 100*(1 - err*1./N) 

print "\nTraining RNN for POS"
# EPOCH LOOP
for i in range(n_iter):

    # SENTENCE LOOP
    cost = 0
    for n, seq in enumerate(train_seq):
        cost += batch_update(seq.x, seq.y)
        # INFO
        perc  = (n+1)*100./len(train_seq) 
        sys.stdout.write("\r%2.2f %%" % perc)
        sys.stdout.flush()

    # Accuracy on train and dev set
    Acc = accuracy(train_seq)
    print "\rEpoch %d: Train cost %2.2f Acc %2.2f %%" % (i+1, cost, Acc),
    print " Devel Acc %2.2f %%" % accuracy(dev_seq)

# Final accuracy on the dev set
print "Test Acc %2.2f %%" % accuracy(test_seq)

\end{python}

\end{exercise}

\section{More Complex RNN the LSTM}


\begin{exercise}

Convince yourself that LSTMs and GRUs are just slightly more complex RNNs
\textbf{TODO}: Use here those nice pics from the blog-post

\begin{python}
# Define the LSTM
lstm = rnns.LSTM(E, n_hidd, n_tags)
# Forward
_p_y = lstm._forward(_x)
# Train cost
_F   = -T.mean(T.log(_p_y)[T.arange(_y.shape[0]), _y]) 
# Total prediction error 
_err = T.sum(T.neq(T.argmax(_p_y,1), _y))

# SGD UPDATE RULE
updates = [(_par, _par - lrate*T.grad(_F, _par)) for _par in lstm.param] 

# COMPILE ERROR FUNCTION, BATCH UPDATE
err_sum      = theano.function([_x, _y], _err)
batch_update = theano.function([_x, _y], _F, updates=updates)

print "\nTraining LSTM for POS"
# EPOCH LOOP
for i in range(n_iter):
    cost = 0
    for n, seq in enumerate(train_seq):
        cost += batch_update(seq.x, seq.y)
        # INFO
        perc  = (n+1)*100./len(train_seq) 
        sys.stdout.write("\r%2.2f %%" % perc)
        sys.stdout.flush()

    # Accuracy on train and dev set
    Acc = accuracy(train_seq)
    print "\rEpoch %d: Train cost %2.2f Acc %2.2f %%" % (i+1, cost, Acc),
    print " Devel Acc %2.2f %%" % accuracy(dev_seq)

# Final accuracy on the dev set
print "Test Acc %2.2f %%" % accuracy(test_seq)
\end{python}
\end{exercise}




