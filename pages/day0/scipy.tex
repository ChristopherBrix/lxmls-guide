



\subsection{Numpy and Matplotlib}

%As discussed in the lecture, there are many helpful functions in numpy. For basic mathematical operations, we have \code{np.log}, \code{np.exp}, \code{np.cos},\ldots with the expected meaning. These operate both on single arguments and on arrays (where they will behave element wise).

%\begin{python}
%import matplotlib.pyplot as plt
%import numpy as np
%X = np.linspace(0, 4 * np.pi, 1000)
%C = np.cos(X)
%S = np.sin(X)

%plt.plot(X, C)
%plt.plot(X, S)
%\end{python}

%Other functions take a whole array and compute a single value from it. For
%example, \code{np.sum}, \code{np.mean},\ldots These are available as both free
%functions and as methods on arrays.

%\begin{python}
%import numpy as np
%A = np.arange(100)
%print np.mean(A)
%print A.mean()

%C = np.cos(A)
%print C.ptp()
%\end{python}

%\begin{exercise}
%Run the above example and lookup the \code{ptp} function/method.
%\end{exercise}

%\begin{exercise}
%Consider the following approximation to compute an integral

%\[
%\int_0^{1} f(x)dx \approx \sum_{i = 0}^{999} \frac{f(i/1000)}{1000}.
%\]

%Use numpy to implement this for $f(x) = x^2$. You should not need to use any
%loops.
%\end{exercise}

\begin{exercise}
\begin{enumerate}
\item Consider the function $f(x) = (x+2)^2 - 16 \exp\left( -(x-2)^2 \right)$.
Draw a plot around the $x \in [-8,8]$ region.
\item What is $\pd{f}{x}$?
\item Use gradient descent to find a local minimum starting from $x_0 = -4$ and
$x_0 = +4$, with $\eta = .01$. Plot all of the intermediate estimates that you
obtain in the same plot.
\end{enumerate}
\begin{python}
import numpy as np
import matplotlib.pyplot as plt
X = np.linspace(-8, 8, 1000)
Y = (X+2)**2 - 16*np.exp(-((X-2)**2))

# derivative of the function f
def get_Y_dev(x):
    return (2*x+4)-16*(-2*x + 4)*np.exp(-((x-2)**2))

def grad_desc(start_x, eps, prec):
    '''
    runs the gradient descent algorithm and returns the list of estimates

    example of use grad_desc(start_x=3.9, eps=0.01, prec=0.00001)
    '''
    x_new = start_x
    x_old = start_x + prec * 2
    res = [x_new]
    while abs(x_old-x_new) > prec:
        x_old = x_new
        x_new = x_old - eps * get_Y_dev(x_new)
        res.append(x_new)
    return np.array(res)
\end{python}
\end{exercise}

Over the next couple of exercises we will make use of the Galton dataset, a
dataset of heights of fathers and sons from the 1877 paper that first discussed
the ``regression to the mean'' phenomenon.

\begin{exercise}
\begin{itemize}
\item Use the \texttt{load()} function in the \texttt{galton.py} file to load
the dataset.
\item What are the mean height and standard deviation of all the people in the
sample? What is the mean height of the fathers and of the sons?
\item Plot a histogram of all the heights (you might want to use the
\texttt{plt.hist} function and the \texttt{ravel} method on arrays).
\item Plot the height of the father versus the height of the son.
\item You should notice that there are several points that are exactly the same
(e.g., there are 21 pairs with the values 68.5 and 70.2). Use the \texttt{?}
command in ipython to read the documentation for the \texttt{numpy.random.rand}
function and add random jitter (i.e., move the point a little bit) to the
points before displaying them. Does your impression of the data change?
\end{itemize}
\end{exercise}

\begin{exercise}
Consider the linear regression problem (ordinary least squares), with a
single response variable

\[
y = x^T w + \varepsilon
\].

The \emph{linear regression problem} is, given a set $\{ y^{(i)} \}_i$ of
samples of $y$ and the corresponding $\vect{x}^{(i)}$ vectors, estimate
$\vect{w}$ to minimise the sum of the $\varepsilon$ variables. Traditionally
this is solved analytically to obtain a closed form solution (although this is
\textbf{not the way in which it should be computed} in this exercise, linear algebra packages
have an optimised solver, with numpy, use \code{numpy.linalg.lstsq}).

Alternatively, we can define the error function for each possible $\vect{w}$:

\[
e(\vect{w}) = \sum_i \left( {\vect{x}^{(i)}}^T \vect{w} - y^{(i)} \right)^2.
\]

\begin{enumerate}
\item Derive the gradient of the error $\pd{e}{w_j}$.
\item Implement a solver based on this for two dimensional problems (i.e.,
$\vect{w} \in R^2$).
\item Use this method on the Galton dataset from the previous exercise to
estimate the relationship between father and son's height. Try two formulas
\begin{equation}
s = f w_1 + \varepsilon,
\label{}
\end{equation}
where $s$ is the son's height, and $f$ is the father heights; and
\begin{equation}
s = f w_1 + 1w_0 + \varepsilon
\label{}
\end{equation}
where the input variable is now two dimensional: $(f,1)$. This allows the
intercept to be non-zero.
\item Plot the regression line you obtain with the points from the previous
exercise.
\item Use the \texttt{np.linalg.lstsq} function and compare to your solution.
\end{enumerate}
\end{exercise}

\subsection{Debugging}


\begin{exercise}
Use the debugger to debug the \texttt{buggy.py} script which attempts to
repeatedly perform the following computation:

\begin{enumerate}
\item Start $x_0 = 0$
\item Iterate

\begin{enumerate}
\item $x'_{t+1} = x_t + r$, where $r$ is a random variable.
\item if $x'_{t+1} >= 1.$, then stop.
\item if $x'_{t+1} <= 0.$, then $x_{t+1} = 0$
\item else $x_{t+1} = x'_{t+1}$.
\end{enumerate}
\item Return the number of iterations.
\end{enumerate}

Having repeated this computation a number of times, the programme prints the
average. Unfortunately, the program has a few bugs, which you need to fix.
\end{exercise}

