\section*{Today's Assignment}
In the previous lesson, you learned the fundamentals of MapReduce and applied
it to a simple classification problem (language detection, using the Na\"{i}ve
Bayes classifier).

Today, we're going to use MapReduce again to solve a trickier
problem: using EM to perform unsupervised POS induction.

\emph{Use the same login information you used yesterday to access your Amazon
machine.}

\section{Distributed EM}

Before you read this section, if you haven't already, please read section
\ref{unsupervised} about the non-distributed version of EM. If necessary,
review that part of the guide, especially the pseudocode in Algorithm
\ref{alg::em}.

EM is an iterative method: for $T$ iterations, we have to alternate between the
E-Step and the M-Step. Both steps can be done in a distributed manner; in this
lesson, we're going to focus on a simple way to distribute the E-Step; the
M-Step will be non-distributed (at first). To see how to distribute both steps
in various configurations see, \emph{e.g.}, \cite{Wolfe2008}.

How can we distribute the E-Step? Recall from equations
\eqref{eq::initialCountsPost}--\eqref{eq::emissionCountsPost} that the E-Step
involves \emph{summing} over $m$, where $m$ indexes each datum in your dataset.
In POS induction, $m$ indexes every sentence. The important factor to
understand is that each sequence can be processed completely independently of
each other (these are the inner expressions) \emph{given the previous
iterations' $\theta^t$}. This will correspond to a map step. The sums over $m$
will be the \emph{reduce} step.

In the last lesson, you already saw the Word Count and Na\"{i}ve Bayes
algorithms. In these two examples, we counted something in the Map step and
then summed them in the Reduce step. That is the intuition behind what we will
do today: we will distribute our data to several workers in the map step, count
things separately, then sum those counts in the reduce step. Everything else
will be exactly the same as in section \ref{unsupervised}.

The most important concepts that you should get out of today's tutorial are:

\begin{enumerate}
\item Since we can process each sequence independently, this is naturally a
\emph{map} step.
\item A full Map/Reduce call will correspond to a single iteration of the
algorithm.
\item To run multiple iterations, you will need to run multiple MapReduce
calls.
\item For each call after the first, you will need to pass in the results of
the previous call.
\end{enumerate}

This is a step up from yesterday's mode of working where a single Map/Reduce
call would solve your problem.

The structure of this lab tutorial is as follows:

\begin{enumerate}
\item You will first look at the code we already provide for you.
\item Based on it, we will run a version that is not distributed.
\item We will convert this to a distributed version that is na\"{i}ve and very inefficient.
\item We will then improve this version to be more efficient.
\item Finally, we will run this code on a large dataset.
\end{enumerate}

\section{First MapReduce Version}

In your amazon machine you will find the directory \texttt{distributed\_em}. In
it you will find a few data and code files:

\begin{itemize}
\item \verb+sequences100.txt+ contains the first 100~sequences.
\item \verb+word_dict_mapping.pkl+ is an auxiliary file.
\item \verb+parallel_em.py+ contains a skeleton of the code we will be writing.
\item A few other files we will use below (but ignore for the moment).
\end{itemize}

\subsection{Loading data}

The datafiles are in a different format from what was used in the previous EM
tutorial \emph{because the simplest way to use Hadoop is make each mapper
receive a single line of the file}.

To load a single line of the input file, we have provided a function
\verb+load_sequence(line, word_dict, tag_dict, mapping)+., which takes the
input data and the metadata in the auxiliary file. Here is how you can use it
to just count the number of tokens in the dataset:

\begin{python}
import pickle
word_dict, tag_dict, mapping = pickle.load(open('word_dict_mapping.pkl'))

n = 0
for line in open('sequences100.txt'):
    s = load_sequence(line, word_dict, tag_dict, mapping)
    n += len(s)
print 'Nr of tokens: ', n
\end{python}

\subsection{Processing a single sequence}

We have also provided code for you to compute the statistics for a single
sequence in the function \verb+sequence_statistics(seq, hmm)+. Here is how you
could use it. First we need to initialize an HMM object:

\begin{python}
import lxmls.sequences.hmm as hmmc
import pickle
word_dict, tag_dict, mapping = pickle.load(open('word_tag_dict_mapping.pkl'))
hmm = hmmc.HMM(word_dict, tag_dict)
hmm.initialize_random()
\end{python}

Here we also made use of the word\_dict\_mapping.pkl file to get the
dictionaries. Then, we initialized the HMM with a random initialization.

Now, we can use the \verb+sequence_statistics+ function:

\begin{python}
for line in open('sequences100.txt'):
    s = load_sequence(line, word_dict, tag_dict, mapping)
    statistics = sequence_statistics(seq, hmm)
\end{python}

It is important to note that \verb+sequence_statistics+ is \emph{a pure
function}! It does not change its inputs. This means that \emph{if you call it
in a different order or in different machines, you will always get the same
results}.

This also explains why we need to have the word dictionaries precomputed: if we
built them as we go, then the order in which the sequences are processed would
make a difference. Thus, we could not process the sequences in parallel. We
will comment on this later when we see an alternative (more sofisticated
implementation). In our case, we were able to compute the dictionaries using a
simple Python script, but if the data was truly large, we could have written
another MapReduce job to discover all the words in the dataset.

If you look back to equations,
\eqref{eq::initialCountsPost}--\eqref{eq::emissionCountsPost} you can see that
\verb+sequence_statistics+ is computing the value inside the outer sums.

\subsection{Combining Partial Results}
At this point, you should understand the code we provided.

The goal of the next exercise is to write a function called \verb+sum_partials+
that can take all the sequence statistics and output the final statistics.

\begin{python}
hmm = \ldots
statistics = []
for line in open('sequences100.txt'):
    s = load_sequence(line, word_dict, tag_dict)
    statistics.append(
        sequence_statistics(seq, hmm)
        )

final = sum_partials(statistics, hmm)
\end{python}

\begin{exercise}
Write the function \verb+sum_partials+. This function should take a list of all
the statistics and an HMM object. It should modify the HMM object to reflect
the results of all the sequences.

Your function should perform some computations and then assign to the hmm object:

\begin{python}
def sum_partials(statistics, hmm):
    hmm.log_likelihood = \ldots
\end{python}

A template is provided in \verb+parallel_em.py+
\end{exercise}

Note that this separation into sequence statistics and combination does not
really correspond to the expectation and maximisation steps.

\subsection{Using MapReduce}
The previous exercise resulted in code that was not distributed, but already
had the map/reduce structure we needed to make it work. The reduce step is not
distributed, but will run on a single machine (this will be improved on in the
next exercises).

We will perform a complete mapreduce run for each iteration of EM that we
compute. In order to run multiple iterations, we will run mapreduce repeatedly.

In what follows, we will save our output to a file called \verb+hmm.pkl+ on
each iteration and then load it from there on the next iteration. Naturally,
the first iteration needs to be a special case and we initialize randomly that
time.

We will also output a Python object from our reduce method. For this, we need a
bit of magic (which is filled in the template we provide and now explain):

\begin{python}
class EMStep(MRJob):
    INTERNAL_PROTOCOL   = PickleProtocol
    OUTPUT_PROTOCOL     = PickleValueProtocol

    def reducer(self, _, partials):
        ...
        yield 'result', a_python_object
\end{python}

By declaring our output protocol to be \verb+PickleValueProtocol+, this means
that we can emit a Python object in the reduce as the final output and it will
be properly serialized to the output.

\begin{exercise}
Based on your function \verb+sum_partials+ and the code we provided you, fill in
the map and reduce steps.

For the moment, your reduce function should have a single emission of the form
\verb+yield 'result', hmm+. Later, we will see more sofisticated methods.

You may want to read the next few paragraphs before you start.
\end{exercise}

In order to test this code on the cluster, you need to run it with the
following flags (we are also directing the output to the file
\verb+next_iteration.pkl+):

\begin{verbatim}
python emjob.py \
    -r local \
    --file=word_dict_mapping.pkl \
    encoded200.txt > next_iteration.pkl
\end{verbatim}

The first flag (\verb+-r local+) makes the code run locally, the second
(\verb+--file=word_dict_mapping.pkl+) declares that the file
\verb+word_dict_mapping.pkl+ is needed to run the job.

In order to run multiple iterations, you need to save the results of the
previous iteration to a file, which can be loaded at startup.

So, once you have run the code the first time, rename the output file:

\begin{verbatim}
mv next_iteration.pkl hmm.pkl
\end{verbatim}

The code should check whether the file \verb+hmm.pkl+ exists and load it if so.
Change your initialization code to read like this:

\begin{verbatim}
from os import path
if path.exists('hmm.pkl'):
    hmm = pickle.load(open('hmm.pkl').read().decode('string-escape'))
else:
    # This is as before
    hmm = hmmc.HMM(word_dict, tag_dict)
    hmm.initialize_random()
\end{verbatim}

We needed to be careful with the string escapes in the above code, but
otherwise were able to just rely on Python \verb+pickle+ to load the results.

And you can now call it again (\emph{passing the hmm.pkl file on the command
line}!):

\begin{verbatim}
python emjob.py \
    -r local \
    --file=word_dict_mapping.pkl \
    --file=hmm.pkl
    encoded200.txt > next_iteration.pkl
\end{verbatim}

\textbf{Pitfall:} Be careful to not over-write your HMM file! You need to
perform two steps:

\begin{enumerate}
\item Run the code, outputing to a temporary file.
\item Rename the temporary file.
\end{enumerate}

The \emph{following is a mistake!}
\begin{verbatim}
python emjob.py \
    -r local \
    --file=word_dict_mapping.pkl \
    --file=hmm.pkl \
    encoded200.txt > hmm.pkl
\end{verbatim}

This is a mistake because the \verb+hmm.pkl+ is now removed (to make space for
the new output) before your program has a chance to read it!

\subsection{Running it on the cluster}

When you are happy with your code, you can now run it on the cluster:

\begin{verbatim}
python emjob.py \
    -r emr \
    --file=word_dict_mapping.pkl \
    encoded200.txt
\end{verbatim}

The \verb+-r emr+ flags runs the code on Elastic MapReduce now. Run it multiple
times using \verb+--file=hmm.pkl+ to get the next iteration.

\emph{This will take a few minutes per iteration, so do not run many
iterations. We will be working on making the code faster before you can do
so.}

\subsubsection{A note on dependencies}

This code depends on packages such as numpy or the lxmls package (which
contains the implementation of the HMM we are using and which you helped write
on the lab tutorial of day 2).

On the cluster we provide you, they are already installed, and you can just use
them without any special effort. If you are attempting to use your own computer
or reproduce the steps at thome; please note that these packages need to be
installed before hand. Check the documentation of mrjob on how to install
python packages on the Elastic MapReduce cluster if you want to run it on your
own Amazon account.

\subsection{Introducing \texttt{mapper\_final}}

While the code you wrote in the last exercise works correctly, it is horribly
inefficient. The problems is that \emph{we output several matrices for each
sequence}. This results in too much communication between processes and the
reduce step needs to load all these large matrices.

In order to address this problem we need to introduce a new operation, the
\verb+map_final+. This called, for each mapper, when all the input has been
processed.

You can perhaps understand this by imagining that mrjob is running the
following loop (in Python pseudo-code):

\begin{python}
for key,value in input:
    job.mapper(key,value)
job.map_final()
\end{python}

You can delay emission of partial results until the \verb+map_final+ step and
then emit the partial output of every sequence this job processed.

For example, here is how WordCount can be performed using \verb+map_final+:

\begin{python}
class WordCount(MRJob):
    def __(init)__:
        self.counts = {}
    
    def mapper(self, _, values):
        for word in values.split():
            if word in self.counts:
                self.counts[word] += 1
            else:
                self.counts[word] = 1

    def map_final(self):
        for k in self.counts:
            yield k, self.counts[k]

    def reducer(self, key, values):
        yield key, sum(values)
\end{python}

Note that the reduce method is still necessary: although each map job will emit
the final result for all the documents it saw, we still need to combine the
results from different map jobs (which run on different machines and processed
different documents).

By only emitting results in the \verb+map_final+ method, we have converted the
implementation to use a batch system: the dataset is partioned into a block for
each processor, each processor works on that whole block, and the reduction is
only made to combine the results of different processors.

This cuts down the communication overhead drastically and also makes the reduce
function be faster and less resource intensive. Now, it only needs to work with
a single set of statistics per compute node instead of one for each input
sequence. Resource usage for reduction now only depends on the number of
machines used and not the size of the input.

\begin{exercise}
Use \verb+map_final+ to improve your MapReduce implementation. Initialize the
matrices and log likelihood to zero in the constructor and update partial sums
in the map function. Emit only in the \verb+map_final+ method.

The reduce function should not need to be changed at all.

Use the previous naïve version as a benchmark. The results should not change
beyond a rounding error (they may change slightly).
\end{exercise}

The implementation that you have now can now be run on a larger dataset. As we
did for yesterday's lab, we have put the full datasets on S3 (you can read back
on S3 on yesterday's guide). You can now test you code on the cluster using the
full dataset:

\begin{verbatim}
python emjob.py \
    -r emr \
    --file=word_dict_mapping.pkl \
    s3://lxmls-labs/all-sequences-for-em.txt
\end{verbatim}

If you try this with your previous version, which did not use the mapper\_final
optimization; you will simply wait a long time before running out of
resources. With this new version, it runs to completion. It might still take
too long for you to be able to run many iterations before the lab is over, but
now you can see that you could run this on a million sequences in a few hours
if you had enough machines. See the note on Hadoop latency and Big Data at the
end of this chapter.

\subsection{Parallelizing Reduce*}

\emph{If you ran out of time to complete this next section, don't worry, this
section is an advanced module and you have already seen the major take-home
points of the tutorial with the previous exercise.}

Our code can still be improved in two ways: (1) there is a single reduce call,
which does not take advantage of the fact that we havea cluster of machines;
and (2) because the emission matrices are sparse we are emitting large matrices
with many zeros. %% FIXME Actually, check if they are still sparse using the map final method

The previous code also had reduce use resources that grow with the number of
processes. This may still be too much if you use a thousand of machines. We
will also avoid this problem.

In order to parallelize reduce, we need to start emitting partial results at a
more fine grained level. Here are the types of emissions we want to consider:

\begin{enumerate}
\item The \textbf{initial counts} this is per state.
\item The \textbf{final counts} this is per state again.
\item The \textbf{transition counts} this is for each pair of states (transition from A to B).
\item The \textbf{emission counts} this is for each pair of state and word.
\item The \textbf{log likelihood}.
\end{enumerate}

In order to distinguish all of these, we will emit them as numbers with
different keys. For example, the log likelihood will simply have the key ``log
likelihood'', while the matrix will have keys which identify the matrix and the
cell.

The code below exemplifies how we could emit the log likelihood and the vector
of final counts:

\begin{python}
yield 'log likelihood', log_likelihood

# Assume that final is a vector of counts
for i in xrange(len(counts)):
    name = hmm.get_state_name(i)
    yield 'final '+name, counts[i]
\end{python} % FIXME does the function get_state_name exist?

For the transition counts and emission counts, we will need a nested for loop:

\begin{python}
for i in xrange(transition.shape[0]):
    for j in xrange(transition.shape[1]):
        if transition[i,j]: # <----- IGNORE ZEROS
            name_i = hmm.get_state_name(i)
            name_j = hmm.get_state_name(j)
            yield 'transition %s %s' % (name_i, name_j), transition[i,j]
\end{python}

The check for zeros is important as it is useless to output zero counts. If the
universe of words is large, then those matrices will be sparse (mostly zeros)
and we avoid extraneous computation.

\begin{exercise}
Look in the file \verb+emission_snippets.py+. This contains the for loops above
and more.

Use these to improve your \verb+map_final+ function. Write the corresponding
reduce function. Do not change the keys used for the emission as they are
needed below (see the next paragraph after this exercise).

\textbf{Important:} Now you should remove the \verb+OUTPUT_PROTOCOL+
declaration! See the note above on what this means.
\end{exercise}

The resulting output is no longer a single HMM object which we can load from
Python, but a text file which encodes all the matrices. If you have used our
snippets exactly, you can also use the code in the function
\verb+parse_hmm_from_output+ to parse this file and generate a new HMM object.

Note that with this output method, we do not need to have the word dictionaries
precomputed. Whereas previously, we relied on matrix indeces to keep our words
apart, now we output the actual word and therefore, we could just process each
sequence as it comes.

\subsection{A Note on Hadoop Overhead and Big Data}

As you probably noticed, Hadoop has a lot of overhead and each iteration takes
a long time to start computing and finish running. In our case, this is a very
significant part of the time it takes to run an iteration.

Hadoop is heavy machinery: it takes a while to move, but then can be very
powerful. The advantages are in the scaling: we had a trillion sequences which
would take thousands of computing hours to process, then the minute or so that
it takes to start up would not matter and we would reap the benefits of working
with hundreds, even thousands, of machines. We can say that Hadoop has high
latency but can have high throughput as well. Thus, it is not very appropriate
for small problems, but can scale to huge ones.

Unfortunately, we only had a few hours in which you could work: including
understanding the task, writing the code, debugging it and running it.
Therefore, it was unfeasible to ask you to work on a problem with a million
sequences which could only be tackled with the heavy machinery of Hadoop.
We could not really work on very large problems. This was only a demo of what
Big Data really is.

However, the code you wrote at the end of the chapter is now perfectly scalable
to any size project you want to tackle. The skills you learned can be applied
to any web-scale problem.

