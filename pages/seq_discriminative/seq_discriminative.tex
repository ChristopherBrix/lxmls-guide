In this class, we will continue to focus on sequence classification,
but instead of following a \emph{generative} approach (like in the previous
chapter) we move towards \emph{discriminative} approaches. Recall that the difference between these approaches is that generative approaches attempt to model the probability distribution of the data, $P(X,Y)$, whereas discriminative ones only model the conditional probability of the sequence, given the observed data, $P(Y|X)$.



\section*{Today's Assignment}

The assignment of today's class is to implement the structured version of the perceptron algorithm.

\section{Classification vs Sequential Classification}


Table \ref{disc_seq_summary} shows how the models for classification 
have counterparts in \emph{sequential} classification. In fact, in
the last chapter we discussed the Hidden Markov model, which can be seen as a
generalization of the Na\"{i}ve Bayes model for sequences. 
In this chapter, we will see a generalization of the
Perceptron algorithm for sequence problems (yielding the Structured
Perceptron, \citealt{collins2002discriminative}) and a generalization of
Maximum Entropy model for sequences (yielding Conditional Random Fields, \citealt{lafferty2001conditional}). 
Note that both these generalizations are  not specific for sequences
and can be applied to a wide range of models (we will see in tomorrow's
lecture how these methods can be applied to parsing).
Although we will not cover all the methods described in
Chapter \ref{day:classification}, bear in mind that all of those have a structured counterpart. 
It should be intuitive after this lecture how those methods could be
extended to structured problems, given the perceptron example.
%Before we explain the particular methods, the next section will talk a bit about feature representation for sequences. 

\begin{table}[h!]
\centering
\begin{tabular}{|l|l|}
\hline
\multicolumn{1}{|c|}{\textbf{Classification}} & \multicolumn{1}{|c|}{\textbf{Sequences}} \\
\hline
\multicolumn{2}{|c|}{\emph{Generative}}\\
\hline
Na\"{i}ve Bayes~(Sec.~\ref{s::naiveBayes}) & Hidden Markov Models~(Sec.~\ref{hmm}) \\
\hline
\multicolumn{2}{|c|}{\emph{Discriminative}}\\
\hline
Perceptron~(Sec.~\ref{s:perceptron}) & Structured Perceptron~(Sec.~\ref{s:spercetron})\\
\hline
Maximum Entropy~(Sec.~\ref{s:me}) & Conditional Random Fields~(Sec.~\ref{s:crf})\\
\hline
\end{tabular}
\caption{\label{disc_seq_summary}Summary of the methods used for classification and sequential classification covered in this guide.}
\end{table}


Throughout this chapter, we will be searching for the solution of 
\begin{equation}
\argmax_{y \in \statevocab^N} P(Y=y | X=x) = \argmax_{y \in \statevocab^N} \W \cdot  \F(x,y). \label{eq::struc_pred} 
\end{equation}
where $\W$ is a weight vector, and $\F(x,y)$ is a feature vector. We will see that this sort of linear models are more flexible than HMMs, in the sense that they may incorporate 
more general features while being able to reuse the same decoders (based on the Viterbi and forward-backward algorithms). In fact, \emph{the exercises in this lab will not touch the decoders that 
have been developed in the previous lab}. \todo{nao percebo muito bem esta nota em italico... Ou ? "in fact, .. will touch" ou "However, ... will touch."} Only the training algorithms and the function that 
compute the scores will change.

As in the previous section, $y = y_1\ldots y_N$ is a sequence so the maximization
is over an exponential number of objects, making it intractable for brute force methods. Again
we will make an assumption analogous to the ``first order Markov independence property,'' so that the
features may decompose as a sum over nodes and edges in a trellis. 
This is done by assuming that expression~\ref{eq::struc_pred} can be written as:
\begin{equation}
\argmax_{y \in \statevocab^N} 
\sum_{i=1}^N \underbrace{\W \cdot \F_{\mathrm{emiss}}(i,x,y_i)}_{\mathrm{score}_{\mathrm{emiss}}}  + 
\underbrace{\W \cdot \F_{\mathrm{init}}(x,y_1)}_{\mathrm{score}_{\mathrm{init}}}  + 
\sum_{i=1}^{N-1} \underbrace{\W \cdot \F_{\mathrm{trans}}(i,x,y_i,y_{i+1})}_{\mathrm{score}_{\mathrm{trans}}}+ 
\underbrace{\W \cdot \F_{\mathrm{final}}(x,y_N)}_{\mathrm{score}_{\mathrm{final}}}
\label{eq::struc_pred_decompose}
\end{equation}
In other words, the scores ${\mathrm{score}_{\mathrm{emiss}}}$, 
${\mathrm{score}_{\mathrm{init}}}$, ${\mathrm{score}_{\mathrm{trans}}}$
and ${\mathrm{score}_{\mathrm{final}}}$ are now computed as inner products 
between weight vectors and feature vectors rather than log-probabilities.
The feature vectors depend locally on the output variable 
(\emph{i.e.}, they only look at a single $y_i$ or a pair $y_i,y_{i+1}$);
however they may depend globally on the observed input $x=x_{1},\ldots,x_{N}$.


\section{\label{seq::features} Features}

In this section we will define two simple feature sets.\footnote{Although not required, all the features we will use are binary features, indicating whether a given condition 
is satisfied or not.} The first feature set
will only use identity features, and will mimic the features used by
the HMM model from the previous section. This will allow us to directly
compare the performance of a generative vs a discriminative
approach. %\begin{example}
%Simple ID Feature set containing the same features as an HMM model.
%\begin{python}
%0 Ms./NOUN NF: id:Ms.::NOUN init_tag:NOUN 
%1 Haag/NOUN NF: id:Haag::NOUN EF: prev_tag:NOUN::NOUN 
%2 plays/VERB NF: id:plays::VERB EF: prev_tag:NOUN::VERB 
%3 Elianti/NOUN NF: id:Elianti::NOUN EF: prev_tag:VERB::NOUN 
%4 ./. NF: id:.::. EF: last_prev_tag:NOUN::. 
%\end{python}
%\end{example}
Table~\ref{id-features} depicts the features that are implicit in the HMM, which was the subject of 
the previous chapter. These features are indicators of initial, transition, final, and emission events.

\begin{table}[h!]
\begin{center}
\begin{tabular}{|l|l|}
\hline
Condition & Name\\
\hline
$y_i = c_k \text{  } \& \text{  } i =0 $& Initial Features \\
\hline
$y_i = c_k \text{  } \& \text{  } y_{i-1} = c_l$& Transition Features \\
\hline
$y_i = c_k \text{  } \& \text{  } i = N$& Final Features \\
\hline
$x_i = w_j \text{  } \& \text{  } y_i = c_k$ & Emission Features \\
\hline
\end{tabular}
\caption{\label{id-features} {\tt IDFeatures} feature set. This set
  replicates the features used by the HMM model.}
\end{center}
\end{table}
 
 
Note that the fact that we were using a generative model has forced us (in some sense) to 
make strong independence assumptions. 
However, since we now move to a discriminative approach, where we model $P(Y|X)$ rather than $P(X,Y)$, we are not tied anymore to 
some of these assumptions. In particular: 
\begin{itemize}
\item We may use ``overlapping'' features, \emph{e.g.}, features that fire simultaneously for many instances. 
For example, we can use a feature for a word, such as a feature which fires for the word "brilliantly", and another for prefixes and suffixes of that word, such as one which fires if the last two letters of the word are "ly".
This would lead to an awkward model if we wanted to insist on a generative approach. 
\item We may use features that depend arbitrarily on the \emph{entire input sequence} $x$. On the other hand, 
we still need to resort to ``local'' features with respect to the \emph{outputs} (\emph{e.g.} looking only at consecutive state pairs), 
otherwise decoding algorithms will become more expensive.  
\todo{NOTA-MA: Nos HMM (de order superior) tb, certo?}
\end{itemize}


This leads us to the second feature set, composed of features that are traditionally used in POS tagging with discriminative models. See Table~\ref{ex-features} for some examples.
Of course, we could have much more complex features, looking arbitrarily to 
the input sequence. We are not going to have them in this
exercise only for performance reasons (to have less features and smaller caches). State-of-the-art sequence classifiers can easily reach over one million features!

\begin{table}[h!]
\begin{center}
\begin{tabular}{|l|l|}
\hline
Condition & Name\\
\hline
$y_i = c_k \text{  } \& \text{  } i =0 $& Initial Features \\
\hline
$y_i = c_k \text{  } \& \text{  } y_{i-1} = c_l$& Transition Features \\
\hline
$y_i = c_k \text{  } \& \text{  } i = N$& Final Features \\
\hline
$x_i = w_j \text{  } \& \text{  } y_i = c_k$& Basic Emission Features \\
$x_i = w_j \text{  } \& \text{ $w_j$ is uppercased} \text{  } \& \text{  } y_i = c_k$& Uppercase Features
\\
$x_i = w_j \text{  } \& \text{ $w_j$ contains digit} \text{  } \& \text{  } y_i = c_k$& Digit Features
\\
$x_i = w_j \text{  } \& \text{ $w_j$ contains hyphen} \text{  } \& \text{  } y_i = c_k$& Hyphen Features
\\
$x_i = w_j \text{  } \& \text{  } w_j[0..i] \forall i \in [1,2,3]
\text{  } \& \text{  } y_i = c_k$& Prefix Features \\
$x_i = w_j \text{  } \& \text{  } w_j[|w_j|-i..|w_j|] \forall i \in [1,2,3] \text{  } \& \text{  } y_i = c_k$& Suffix Features \\
\hline
\end{tabular}
\caption{\label{ex-features} {\tt Extended} feature set. Some features in this set could not be included in the HMM model. The features included in the bottom row are all considered emission features 
for the purpose of our implementation, since they all depend on $i$, $x$ and $y_i$.}
\end{center}
\end{table}



%\begin{example}
%
%\begin{python}
%0 Ms./NOUN NF: id:Ms.::NOUN uppercased::NOUN suffix:.::NOUN suffix:s.::NOUN prefix:M::NOUN prefix:Ms::NOUN init_tag:NOUN 
%1 Haag/NOUN NF: id:Haag::NOUN uppercased::NOUN suffix:g::NOUN suffix:ag::NOUN suffix:aag::NOUN prefix:H::NOUN prefix:Ha::NOUN prefix:Haa::NOUN rare::NOUN EF: prev_tag:NOUN::NOUN 
%2 plays/VERB NF: id:plays::VERB EF: prev_tag:NOUN::VERB 
%3 Elianti/NOUN NF: id:Elianti::NOUN uppercased::NOUN suffix:i::NOUN suffix:ti::NOUN suffix:nti::NOUN prefix:E::NOUN prefix:El::NOUN prefix:Eli::NOUN rare::NOUN EF: prev_tag:VERB::NOUN 
%4 ./. NF: id:.::. EF: last_prev_tag:NOUN::. 
%\end{python}
%\end{example}



Our features subdivide into two groups: 
$\F_{\mathrm{emiss}}$, $\F_{\mathrm{init}}$, 
and $\F_{\mathrm{final}}$ are all instances of 
\emph{node features}, 
depending only on a single position in the
state sequence (a node in the trellis); 
$\F_{\mathrm{trans}}$ 
are \emph{edge features},
depending on two consecutive positions in the state 
sequence (an edge in the trellis).
Similarly as in the previous chapter, we have the following scores (also called 
\emph{log-potentials} in the literature on CRFs and graphical models):
\begin{itemize}
\item \emph{Initial scores.}
These are scores for the initial state. 
They are given by 
\begin{equation}
{\mathrm{score}}_{\mathrm{init}}(x,y_1) = \W \cdot \F_{\mathrm{init}}(x,y_1).
\end{equation} 
\item \emph{Transition scores.}
These are scores for two consecutive states at a particular position. 
They are given by 
\begin{equation}
{\mathrm{score}}_{\mathrm{trans}}(i,x,y_i,y_{i+1}) = \W \cdot \F_{\mathrm{trans}}(i,x,y_{i},y_{i+1}).
\end{equation} 
\item \emph{Final scores.}
These are scores for the final state. 
They are given by 
\begin{equation}
{\mathrm{score}}_{\mathrm{final}}(x,y_N) = \W \cdot \F_{\mathrm{final}}(x,y_N).
\end{equation} 
\item \emph{Emission scores.}
These are scores for a state at a particular position. 
They are given by 
\begin{equation}\label{dis_node_potentials}
{\mathrm{score}}_{\mathrm{emiss}}(i,x,y_i) = \W \cdot \F_{\mathrm{emiss}}(i,x,y_i).
\end{equation} 
\end{itemize}
%which form a vector $\boldsymbol{f}_N(\sent,\hv_i)$, 
%and \emph{edge features}, which form a vector $\boldsymbol{f}_E(\sent,\hv_i,\hv_{i-1})$.% 
%\footnote{To make things simpler, we will assume later on that edge features do not depend on the input $\sent$---but they could, without 
%changing at all the decoding algorithm.} %
%These feature vectors will receive parameter vectors $\boldsymbol{\theta}_N$ and  $\boldsymbol{\theta}_E$. 
%Similarly as in the previous chapter, we consider:
%\begin{itemize}
%\item\emph{Node Potentials.} These are scores for a state at a particular position. 
% They are given by 
% \begin{equation}\label{dis_node_potentials}
% \psi_V(\sent,\hs_i) = \exp(\boldsymbol{\theta}_V \cdot \boldsymbol{f}_V(\sent,\hs_i)).
% \end{equation}
%\item\emph{Edge Potentials.} These are scores for the transitions. They are given by 
% \begin{equation}\label{dis_edge_potentials}
%\psi_E(\sent,\hs_i,\hs_{i-1}) = \exp(\boldsymbol{\theta}_E \cdot \boldsymbol{f}_E(\sent,\hs_i,\hs_{i-1})). 
% \end{equation}
%\end{itemize}
%
%Let $\boldsymbol{\theta} = (\boldsymbol{\theta}_N, \boldsymbol{\theta}_E)$. 

\section{Discriminative Sequential Classifiers}

Given a weight vector $\W$, the conditional probability $P_{\W}(Y=y|X=x)$ is then defined as follows: 
\begin{eqnarray}
\lefteqn{P_{\W}(Y=y|X=x) =}\nonumber\\
&&
\!\!\!\!\!\!\!\!\frac{1}{Z({\W},x)}\exp\left(\W \cdot \F_{\mathrm{init}}(x,y_1) + 
\sum_{i=1}^{N-1} \W \cdot \F_{\mathrm{trans}}(i,x,y_i,y_{i+1})
+
\W \cdot \F_{\mathrm{final}}(x,y_N)
+
\sum_{i=1}^{N} \W \cdot \F_{\mathrm{emiss}}(i,x,y_i)
\right) \label{eq:disc_formula}
%&=& \frac{1}{Z(\boldsymbol{\theta},x)} \prod_i\psi_V(\sent_i,\hs_i) \psi_E(\sent_i,\hs_i,\hs_{i-1}),
\end{eqnarray}
where the normalizing factor $Z(\W,x)$ is called the \emph{partition function}:
\begin{eqnarray}
\lefteqn{Z(\W,x) =}\nonumber\\ 
&& 
\sum_{y\in \statevocab^N} \exp\left(\W \cdot \F_{\mathrm{init}}(x,y_1) + 
\sum_{i=1}^{N-1} \W \cdot \F_{\mathrm{trans}}(i,x,y_i,y_{i+1})
+
\W \cdot \F_{\mathrm{final}}(x,y_N)
+
\sum_{i=1}^{N} \W \cdot \F_{\mathrm{emiss}}(i,x,y_i)
\right).
\end{eqnarray}


\subsection{Training}

For training, 
the important problem is that of obtaining the weight vector $\W$ that lead to an accurate 
classifier. 
We will discuss two possible strategies:
\begin{enumerate}
\item Maximizing conditional log-likelihood from a set of labeled data $\{(x^m,y^m)\}_{m=1}^M$, yielding \textbf{conditional random fields}. This corresponds to the following optimization problem:
\begin{equation}
\widehat{\W} = \argmax_{\W} \sum_{m=1}^M \log P_{\W}(Y=y^m | X=x^m).
\end{equation}
To avoid overfitting, it is common to regularize with the Euclidean norm function, 
which is equivalent to considering a zero-mean Gaussian prior on the weight vector.
The problem becomes:
\begin{equation}
\widehat{\W} = \argmax_{\W} \sum_{m=1}^M \log P_{\W}(Y=y^m | X=x^m) - \frac{\lambda}{2} \|\W\|^2.
\end{equation}
This is precisely the structured variant of the logistic regression 
method discussed in Chapter 1.
Unlike HMMs, this problem does not have a closed form solution 
and has to be solved with numerical optimization. 
\item Alternatively, running the \textbf{structured perceptron} algorithm 
to obtain a weight vector $\W$ that accurately
classifies the training data. 
We will see that this simple strategy achieves results which are competitive 
with conditional log-likelihood maximization.
\end{enumerate}

\subsection{Decoding}

For decoding,  
there are three important problems that need to be solved: 
\begin{enumerate}
\item Given $X=x$, compute the most likely output sequence $\widehat{y}$ (the one which maximizes $P_{\W}(Y=y|X=x)$). 
\item Compute the posterior marginals $P_{\W}(Y_i=y_i|X=x)$ at each position $i$.
\item Evaluate the partition function $Z(\W,x)$. 
\end{enumerate}
Interestingly, all these problems can be solved by using the very same
algorithms that were 
already implemented for HMMs: the Viterbi algorithm (for 1) and the forward-backward algorithm (for 2--3). All that changes is the way the scores are computed. 

\section{\label{s:crf}Conditional Random Fields}

Conditional Random Fields (CRF)~\citep{lafferty2001conditional} can be seen
as an extension of the Maximum Entropy (ME) models  to structured problems.%
\footnote{An earlier, less successful, attempt to perform such an extension was via Maximum Entropy Markov
models (MEMM)~\citep{mccallum2000maximum}. There each factor (a node or edge) 
is a \emph{locally} normalized maximum entropy model. A shortcoming of MEMMs is its 
so-called \emph{labeling bias} \citep{Bottou1991}, which makes them biased towards states
with few successor states (see \cite{lafferty2001conditional} for more information).}
They are \emph{globally} normalized models: the probability of a given
sentence is given by Equation \ref{eq:disc_formula}. 
There are only two differences with respect to the standard ME model 
described a couple of days ago for multi-class classification: 
\begin{itemize}
\item Instead of computing the posterior marginals $P(Y=y|X=x)$ for all possible configurations
of the output variables (which are exponentially many), it 
assumes the model decompose into ``parts'' (in this case, nodes and edges), and it computes only 
the posterior marginals for those parts, 
  $P(Y_i=y_i | X=x)$ and  $P(Y_i=y_i, Y_{i+1}=y_{i+1}| X=x)$. 
Crucially, \textbf{we can compute these quantities by using the very same forward-backward algorithm that we have described for HMMs}.
\item Instead of updating the features for all possible outputs $y' \in \Lambda^N$, 
we again exploit the decomposition into parts above and update only 
``local features'' at the nodes and edges.
\end{itemize}

Algorithm
\ref{alg:crf_online} shows the pseudo code to optimize a CRF with 
the stochastic gradient descent (SGD) algorithm  
(our toolkit also includes an implmentation of a quasi-Newton method, L-BFGS, 
which converges faster, but for the purpose of this exercise, we will 
stick with SGD). 

\begin{algorithm}[h!]
   \caption{SGD for Conditional Random Fields \label{alg:crf_online}}
\begin{algorithmic}[1]
   \STATE {\bfseries input:} $\mathcal{D}$, $\lambda$, number of rounds $T$,
   learning rate sequence $(\eta_t)_{t = 1,\ldots,T}$
   \STATE initialize $\W^1 = \mathbf{0}$
	\FOR{$t=1$ {\bfseries to} $T$}
	\STATE choose $m=m(t)$ randomly
	\STATE take training pair $(x^m, y^m)$ and compute the probability 
	$P_{\W}(Y=y^m|X=x^m)$ using the current model ${\W}$ and Eq.~\ref{eq:disc_formula}.
	\STATE for every $i$, $y_i'$, and $y_{i+1}'$, 
	compute marginal probabilities $P(Y_i=y_i' | X=x)$ and  $P(Y_i=y_i', Y_{i+1}=y_{i+1}'| X=x^m)$ 
	using the current model $\W$, for each node and edge, 
        through the forward-backward algorithm.
	\STATE compute the feature vector expectation:  
	\begin{eqnarray}
	E_{\W^t}[\F(x^m,Y)] &=& \sum_{y_1} P(y_1|x^m)\F_{\mathrm{init}}(x^m,y_1) +\nonumber\\
	&& \sum_{i=1}^{N-1}\sum_{y_i,y_{i+1}} P(y_i,y_{i+1}|x^m)\F_{\mathrm{trans}}(i,x^m,y_i,y_{i+1}) +\nonumber\\
	&& \sum_{y_N} P(y_N|x^m)\F_{\mathrm{final}}(x^m,y_N) +\nonumber\\
	&& \sum_{i=1}^{N}\sum_{y_i} P(y_i|x^m)\F_{\mathrm{emiss}}(i,x^m,y_i).
	\end{eqnarray}
	\STATE update the model: 
	$$\W^{t+1} \leftarrow (1-\lambda \eta_t) \boldsymbol{\W}^{t} + \eta_t \left( \F(x^m,y^m) 
	- E_{\W^t}[\F(x^m,Y)]\right)$$
	\ENDFOR
   \STATE \textbf{output:} $\widehat{\W} \leftarrow \W^{T+1}$
\end{algorithmic}
\end{algorithm}


\begin{exercise}\label{exer:crf1}
In this exercise you will train a CRF
using different feature sets for \pos\ Tagging. Start with the code below, which uses the ID feature set from table \ref{id-features}.
\begin{python}
import sequences.crf_online as crfo
import sequences.structured_perceptron as spc
import readers.pos_corpus as pcc
import sequences.id_feature as idfc
import sequences.extended_feature as exfc

print "CRF Exercise"

corpus = pcc.PostagCorpus()
train_seq = corpus.read_sequence_list_conll("../data/train-02-21.conll",max_sent_len=10,max_nr_sent=1000)
test_seq = corpus.read_sequence_list_conll("../data/test-23.conll",max_sent_len=10,max_nr_sent=1000)
dev_seq = corpus.read_sequence_list_conll("../data/dev-22.conll",max_sent_len=10,max_nr_sent=1000)
feature_mapper = idfc.IDFeatures(train_seq)
feature_mapper.build_features()

crf_online = crfo.CRFOnline(corpus.word_dict, corpus.tag_dict, feature_mapper)
crf_online.num_epochs = 20
crf_online.train_supervised(train_seq)

Epoch: 0 Objective value: -5.779018
Epoch: 1 Objective value: -3.192724
Epoch: 2 Objective value: -2.717537
Epoch: 3 Objective value: -2.436614
Epoch: 4 Objective value: -2.240491
Epoch: 5 Objective value: -2.091833
Epoch: 6 Objective value: -1.973353
Epoch: 7 Objective value: -1.875643
Epoch: 8 Objective value: -1.793034
Epoch: 9 Objective value: -1.721857
Epoch: 10 Objective value: -1.659605
Epoch: 11 Objective value: -1.604499
Epoch: 12 Objective value: -1.555229
Epoch: 13 Objective value: -1.510806
Epoch: 14 Objective value: -1.470468
Epoch: 15 Objective value: -1.433612
Epoch: 16 Objective value: -1.399759
Epoch: 17 Objective value: -1.368518
Epoch: 18 Objective value: -1.339566
Epoch: 19 Objective value: -1.312636
\end{python}
\todo{Acho que nao vale a pena ter o output de tantas epocas}

You will receive feedback when each epoch is finished, note that
running the 20 epochs might take a while.

After training is done, evaluate the learned model on the training, development and test sets.
\todo{NOTA-MA: a) O development set n�o � usado. b) � suposto os conjuntos de teste e desenvolvimento terem 1000 frases (e n�o 200)?}

\begin{python}
pred_train = crf_online.viterbi_decode_corpus(train_seq)
pred_dev = crf_online.viterbi_decode_corpus(dev_seq)
pred_test = crf_online.viterbi_decode_corpus(test_seq)

eval_train = crf_online.evaluate_corpus(train_seq, pred_train)
eval_dev = crf_online.evaluate_corpus(dev_seq, pred_dev)
eval_test = crf_online.evaluate_corpus(test_seq, pred_test)

print "CRF - ID Features Accuracy Train: %.3f Dev: %.3f Test: %.3f"%(eval_train,eval_dev,eval_test)
\end{python}

You should get values similar to these:
\begin{python}
Out[]: CRF - 
ID Features Accuracy Train: 0.949 Dev: 0.846 Test: 0.858
\end{python}
\end{exercise}

Compare with the results achieved with the HMM model (0.837 on the test set, from the previous lecture). Even using a similar feature set a CRF yields better
results than the HMM from the previous lecture. 
Perform some error analysis and figure out what are the main
errors the tagger is making. Compare them with the errors made
by the HMM model. (Hint: use the methods developed in the previous
lecture to help you with the error analysis).


\begin{exercise}\label{exer:crf2}
Repeat the previous exercise using the extended feature set. Compare the results.

\begin{python}
feature_mapper = exfc.ExtendedFeatures(train_seq)
feature_mapper.build_features()

crf_online = crfo.CRFOnline(corpus.word_dict, corpus.tag_dict, feature_mapper)
crf_online.num_epochs = 20
crf_online.train_supervised(train_seq)

Epoch: 0 Objective value: -7.141596
Epoch: 1 Objective value: -1.807511
Epoch: 2 Objective value: -1.218877
Epoch: 3 Objective value: -0.955739
Epoch: 4 Objective value: -0.807821
Epoch: 5 Objective value: -0.712858
Epoch: 6 Objective value: -0.647382
Epoch: 7 Objective value: -0.599442
Epoch: 8 Objective value: -0.562584
Epoch: 9 Objective value: -0.533411
Epoch: 10 Objective value: -0.509885
Epoch: 11 Objective value: -0.490548
Epoch: 12 Objective value: -0.474318
Epoch: 13 Objective value: -0.460438
Epoch: 14 Objective value: -0.448389
Epoch: 15 Objective value: -0.437800
Epoch: 16 Objective value: -0.428402
Epoch: 17 Objective value: -0.419990
Epoch: 18 Objective value: -0.412406
Epoch: 19 Objective value: -0.405524

pred_train = crf_online.viterbi_decode_corpus(train_seq)
pred_dev = crf_online.viterbi_decode_corpus(dev_seq)
pred_test = crf_online.viterbi_decode_corpus(test_seq)

eval_train = crf_online.evaluate_corpus(train_seq, pred_train)
eval_dev = crf_online.evaluate_corpus(dev_seq, pred_dev)
eval_test = crf_online.evaluate_corpus(test_seq, pred_test)

print "CRF - Extended Features Accuracy Train: %.3f Dev: %.3f Test: %.3f"%(eval_train,eval_dev,eval_test)
\end{python}

You should get values close to the following:
\begin{python}
CRF - Extended Features Accuracy Train: 0.984 Dev: 0.899 Test: 0.894
\end{python}


Compare the errors obtained with the two different feature
sets. Do some error analysis: what errors were correct by using
more features? Can you think of other features to use to solve the
errors found?
\end{exercise}

The main lesson to learn from this exercise is that, usually, if you are not satisfied by the accuracy of your algorithm, you can perform some error analysis and find out which errors your algorithm is making. You can then add more features which attempt to improve those specific errors (this is known as \emph{feature engineering}). This can lead to two problems:
\begin{itemize}
\item More features will make training and decoding more expensive. For example, if you add features that depend on the current word and the previous word, the number of new features is the square of the number of different words, which is quite large. For example, the Penn Treebank has around 40000 different words, so you are adding a lot of new features, even though not all pairs of words will ever occur. Features that depend on three words (previous, current, and next) are even more numerous.
\item If features are very specific, such as the (previous word, current word, next word) one just mentioned, they might occur very rarely in the training set, which leads to overfit problems. Some of these problems (not all) can be mitigated with techniques such as smoothing, which you already learned about.
\end{itemize}




\section{\label{s:spercetron}Structured Perceptron}

The structured perceptron \citep{collins2002discriminative}, namely its averaged version, is a very simple
algorithm that relies on Viterbi decoding and very simple additive
updates. In practice this algorithm is very easy to implement and
behaves remarkably well in a variety of problems. These two
characteristics make the structured perceptron algorithm a natural
first choice to try and test a new problem or a new feature set. 

Recall what you learned about the
perceptron algorithm (Section \ref{s:perceptron}) and compare it against the structured perceptron
(Algorithm \ref{alg:structured-perceptron}). 

\begin{algorithm}[h!]
   \caption{Averaged Structured perceptron \label{alg:structured-perceptron}}
\begin{algorithmic}[1]
   \STATE {\bfseries input:} $\mathcal{D}$, number of rounds $T$
   \STATE initialize $\W^1 = \mathbf{0}$
	\FOR{$t=1$ {\bfseries to} $T$}
	\STATE choose $m=m(t)$ randomly
	\STATE take training pair $(x^m, y^m)$ and 
predict using the current model $\W$, through the Viterbi algorithm: 
	$$\widehat{y}  \leftarrow \argmax_{y' \in \Lambda^N} \W^t \cdot \F(x^m,y')$$	

	\STATE update the model: 
	$\W^{t+1} \leftarrow \W^{t} + \F(x^m,y^m) - \F(x^m,\widehat{y})$
	\ENDFOR
   \STATE \textbf{output:} the averaged model $\widehat{\W} \leftarrow \frac{1}{T}\sum_{t=1}^T \W^t$
\end{algorithmic}
\end{algorithm}



There are only two differences, which mimic the ones already seen for the comparison between CRFs 
and multi-class ME models:
\begin{itemize}
\item Instead of explicitly enumerating all possible output 
configurations (exponentially many of them) to compute 
 $\widehat{y} := \argmax_{y'\in\mathcal{Y}} \W \cdot \F(x^m,y')$, 
it finds the best sequence through the Viterbi algorithm. 
\item Instead of updating the features for the entire $\widehat{y}$, 
it updates only the node and edge features at the positions where the
  labels are different---i.e., where mistakes are made.
\end{itemize}



\section{Assignment}

\begin{exercise}\label{exer:strucperc1}
Implement the structured perceptron algorithm.\\ 
To do this, edit file {\tt sequences/structured\_perceptron.py} 
and implement the function
\begin{python}
def perceptron_update(self, sequence):
    pass
\end{python}
This function should apply one round of the perceptron algorithm,
updating the weights for a given sequence, and returning
the number of predicted labels (which equals the sequence length) 
and the number of mistaken labels. 

Hint: adapt the function 
\begin{python}
def gradient_update(self, sequence, eta):
\end{python}
defined in file {\tt sequences/crf\_online.py}. 
You will need to replace the computation of posterior marginals 
by the Viterbi algorithm, and to change the parameter updates 
according to Algorithm \ref{alg:structured-perceptron}.
\end{exercise}


\begin{exercise}
Repeat Exercises~\ref{exer:crf1}--\ref{exer:crf2} using the structured perceptron algorithm 
instead of a CRF. Report the results. 

Here is the code for the simple feature set:


\begin{python}
feature_mapper = idfc.IDFeatures(train_seq)
feature_mapper.build_features()

print "Perceptron Exercise"

sp = spc.StructuredPerceptron(corpus.word_dict, corpus.tag_dict, feature_mapper)
sp.num_epochs = 20
sp.train_supervised(train_seq)

Epoch: 0 Accuracy: 0.656806
Epoch: 1 Accuracy: 0.820898
Epoch: 2 Accuracy: 0.879176
Epoch: 3 Accuracy: 0.907432
Epoch: 4 Accuracy: 0.925239
Epoch: 5 Accuracy: 0.939956
Epoch: 6 Accuracy: 0.946284
Epoch: 7 Accuracy: 0.953790
Epoch: 8 Accuracy: 0.958499
Epoch: 9 Accuracy: 0.955114
Epoch: 10 Accuracy: 0.959235
Epoch: 11 Accuracy: 0.968065
Epoch: 12 Accuracy: 0.968212
Epoch: 13 Accuracy: 0.966740
Epoch: 14 Accuracy: 0.971302
Epoch: 15 Accuracy: 0.968653
Epoch: 16 Accuracy: 0.970419
Epoch: 17 Accuracy: 0.971891
Epoch: 18 Accuracy: 0.971744
Epoch: 19 Accuracy: 0.973510

pred_train = sp.viterbi_decode_corpus(train_seq)
pred_dev = sp.viterbi_decode_corpus(dev_seq)
pred_test = sp.viterbi_decode_corpus(test_seq)



eval_train = sp.evaluate_corpus(train_seq, pred_train)
eval_dev = sp.evaluate_corpus(dev_seq, pred_dev)
eval_test = sp.evaluate_corpus(test_seq, pred_test)

print "Structured Perceptron - ID Features Accuracy Train: %.3f Dev: %.3f Test: %.3f"%(eval_train,eval_dev,eval_test)
\end{python}


\begin{python}
Structured Perceptron - ID Features Accuracy Train: 0.984 Dev: 0.835 Test: 0.840
\end{python}

Here is the code for the extended feature set:

\begin{python}
feature_mapper = exfc.ExtendedFeatures(train_seq)
feature_mapper.build_features()
sp = spc.StructuredPerceptron(corpus.word_dict, corpus.tag_dict, feature_mapper)
sp.num_epochs = 20
sp.train_supervised(train_seq)

Epoch: 0 Accuracy: 0.764386
Epoch: 1 Accuracy: 0.872701
Epoch: 2 Accuracy: 0.903458
Epoch: 3 Accuracy: 0.927594
Epoch: 4 Accuracy: 0.938484
Epoch: 5 Accuracy: 0.951141
Epoch: 6 Accuracy: 0.949816
Epoch: 7 Accuracy: 0.959529
Epoch: 8 Accuracy: 0.957616
Epoch: 9 Accuracy: 0.962325
Epoch: 10 Accuracy: 0.961148
Epoch: 11 Accuracy: 0.970567
Epoch: 12 Accuracy: 0.968212
Epoch: 13 Accuracy: 0.973216
Epoch: 14 Accuracy: 0.974393
Epoch: 15 Accuracy: 0.973951
Epoch: 16 Accuracy: 0.976600
Epoch: 17 Accuracy: 0.977483
Epoch: 18 Accuracy: 0.974834
Epoch: 19 Accuracy: 0.977042

pred_train = sp.viterbi_decode_corpus(train_seq)
pred_dev = sp.viterbi_decode_corpus(dev_seq)
pred_test = sp.viterbi_decode_corpus(test_seq)

eval_train = sp.evaluate_corpus(train_seq, pred_train)
eval_dev = sp.evaluate_corpus(dev_seq, pred_dev)
eval_test = sp.evaluate_corpus(test_seq, pred_test)

print "Structured Perceptron - Extended Features Accuracy Train: %.3f Dev: %.3f Test: %.3f"%(eval_train,eval_dev,eval_test)
\end{python}

And here are the expected results:
\todo{NOTA-MA: ainda tenho de terminar o codigo.}
\begin{python}
Structured Perceptron - Extended Features Accuracy Train: 0.984 Dev: 0.888 Test: 0.890
\end{python}

\end{exercise}





%\begin{exercise}\label{exer:strucperc1}
%In this exercise you will test the structured perceptron algorithm
%using different feature sets for \pos\ Tagging. Start with the code below, which uses the ID feature set from table \ref{id-features}.
%\begin{python}
%import sequences.structured_perceptron as spc
%import sequences.crf_batch as crfc
%import readers.pos_corpus as pcc
%import sequences.id_feature as idfc
%import sequences.extended_feature as exfc
%
%
%print "Perceptron Exercise"
%corpus = pcc.PostagCorpus()
%train_seq = corpus.read_sequence_list_conll("../data/train-02-21.conll",max_sent_len=10,max_nr_sent=1000)
%test_seq = corpus.read_sequence_list_conll("../data/test-23.conll",max_sent_len=10,max_nr_sent=1000)
%dev_seq = corpus.read_sequence_list_conll("../data/dev-22.conll",max_sent_len=10,max_nr_sent=1000)
%corpus.add_sequence_list(train_seq) 
%id_f = idfc.IDFeatures(corpus)
%id_f.build_features()
%sp = spc.StructuredPercetron(corpus,id_f)
%sp.nr_rounds = 20
%sp.train_supervised(train_seq.seq_list)
%
%\end{python}
%You will get some messages about "unknown tags", which are a consequence of using a reduced set of 12 POS tags instead of the full set. You will also receive feedback when each epoch is finished.
%
%After training is done, evaluate the learned model on the training, development and test sets.
%\begin{python}
%pred_train = sp.viterbi_decode_corpus(train_seq.seq_list)
%pred_dev = sp.viterbi_decode_corpus(dev_seq.seq_list)
%pred_test = sp.viterbi_decode_corpus(test_seq.seq_list)
%
%eval_train = sp.evaluate_corpus(train_seq.seq_list,pred_train)
%eval_dev = sp.evaluate_corpus(dev_seq.seq_list,pred_dev)
%eval_test = sp.evaluate_corpus(test_seq.seq_list,pred_test)
%
%print "Structured Percetron - ID Features Accuracy Train: %.3f Dev: %.3f Test: %.3f"%(eval_train,eval_dev,eval_test)
%\end{python}
%
%You should get values similar to these:
%\begin{python}
%Out[]: Structured Percetron - ID Features Accuracy Train: 0.972 Dev: 0.799 Test: 0.811
%
%
%\end{python}
%\end{exercise}
%
%%Compare with the results achieved with the HMM model (0.838 on the test set, from the previous lecture). Even using a similar feature set the structured perceptron yields better
%%results than the HMM from the previous lecture. 
%Perform some error analysis and figure out what are the main
%errors the perceptron is making. Compare them with the errors made
%by the HMM model. (Hint: use the methods developed in the previous
%lecture to help you with the error analysis).
%
%
%\begin{exercise}\label{exer:strucperc2}
%Repeat the previous exercise using the extended feature set. Compare the results.
%
%\begin{python}
%ex_f = exfc.ExtendedFeatures(corpus)
%ex_f.build_features()
%sp = spc.StructuredPercetron(corpus,ex_f)
%sp.nr_rounds = 20
%sp.train_supervised(train_seq.seq_list)
%
%pred_train = sp.viterbi_decode_corpus(train_seq.seq_list)
%pred_dev = sp.viterbi_decode_corpus(dev_seq.seq_list)
%pred_test = sp.viterbi_decode_corpus(test_seq.seq_list)
%
%eval_train = sp.evaluate_corpus(train_seq.seq_list,pred_train)
%eval_dev = sp.evaluate_corpus(dev_seq.seq_list,pred_dev)
%eval_test = sp.evaluate_corpus(test_seq.seq_list,pred_test)
%
%print "Structured Percetron - Extended Features Accuracy Train: %.3f Dev: %.3f Test: %.3f"%(eval_train,eval_dev,eval_test)
%
%\end{python}
%
%You should get values close to the following:
%\begin{python}
%Structured Percetron - Extended Features Accuracy Train: 0.973 Dev: 0.887 Test: 0.881
%\end{python}
%
%Compare the errors obtained with the two different feature
%sets. Do some error analysis: what errors were correct by using
%more features? Can you think of other features to use to solve the
%errors found?
%\end{exercise}
%
%The main lesson to learn from this exercise is that, usually, if you are not satisfied by the accuracy of your algorithm, you can perform some error analysis and find out which errors your algorithm is making. You can then add more features which attempt to improve those specific errors (this is known as \emph{feature engineering}). This can lead to two problems:
%\begin{itemize}
%\item More features will make training and decoding more expensive. For example, if you add features that depend on the current word and the previous word, the number of new features is the square of the number of different words, which is quite large. For example, the Penn Treebank has around 40000 different words, so you are adding a lot of new features, even though not all pairs of words will ever occur. Features that depend on three words (previous, current, and next) are even more numerous.
%\item If features are very specific, such as the (previous word, current word, next word) one just mentioned, they might occur very rarely in the training set, which leads to overfitting problems. Some of these problems (not all) can be mitigated with techniques such as smoothing, which you already learned about.
%\end{itemize}



%%% Local Variables: 
%%% mode: latex
%%% TeX-master: "../../guide"
%%% End: 
