%So far we have not committed to any form for the probability
%distributions $\pi_l$, $a_{m,l}$ and $b_l(\obs_i)$. In both applications
%addressed in this class, both the observations and the hidden
%variables are discrete. The most common approach is to model each of
%these probability distributions as multinomial distributions,
%summarized in Table \ref{tt:mult-params}. Note that the number of parameters of $a_{l,m}$ is $|\hvocab|(|\hvocab|+1)$ because of the special ``STOP'' symbol.

%\begin{table}
%\begin{center}
%\begin{tabular}{|c|c|c|c|}
%\hline
%short notation & probability distribution  & |parameters|& constraint \\
%\hline
%$\pi_j$ & $p_{\theta} (\hs_1 = \hv_j)$ & $|\hvocab|$ & $\sum_{\hv \in \hvocab} \pi_j = 1$;\\
%\hline
%$a_{l,m}$ & $p_{\theta} (\hs_i = \hv_l \mid \hs_{i-1} = \hv_m)$ & $|\hvocab|(|\hvocab|+1)$ &$\sum_{\hv_l \in \hvocab} a_{m,l} = 1$;\\
%\hline
%%$f_{l,m}$ & $p_{\theta} (\hs_N = \hv_l \mid \hs_{N-1} = \hv_m)$ & $(|\hvocab|-1)^2$ &$\sum_{\hv_l \in \hvocab} t_{m,l} = 1$;\\
%%\hline
%$b_q(l) $& $p_{\theta}(\obs_i = \vv_q\mid \hs_i = \hv_l)$  & $|\hvocab||\vocab|$ &$\sum_{\vv_q \in \vocab} b_q(l)  = 1$.\\
%\hline
%\end{tabular}
%\end{center}
%\caption[HMM multinomial parametrization]{\label{tt:mult-params}Multinomial parametrization of the HMM distributions.}
%\end{table}

One important problem in HMMs is to estimate the 
model parameters, \emph{i.e.}, 
the distributions depicted in Table~\ref{tab:hmm-dist}. 
We will refer to the set of all these parameters 
as $\theta$. 
In a supervised setting, the HMM model
is trained to maximize the joint log-likelihood of the data. Given a
dataset $\mathcal{D}_L$, the objective being optimized is:
\begin{equation}
\argmax_{\theta} \sum_{m=1}^M \log P_{\theta}(X=x^m,Y=y^m),
\end{equation}
where $P_{\theta}(X=x^m,Y=y^m)$ is given by Eq.~\ref{eqn:hmm}.

In some applications (\emph{e.g.} speech recognition) 
the observation variables are continuous, hence the emission distributions are real-valued (\emph{e.g.} mixtures of Gaussians).
In our case, both the state set and the observation set are discrete (and finite), therefore we use
multinomial distributions for the emission and 
transition probabilities. 
Multinomial distributions are attractive for several reasons: first of
all, they are easy to implement; secondly, the maximum likelihood estimation of the parameters has a simple closed form. The parameters are just
normalized counts of events that occur in the corpus (the same as the
Na\"{i}ve Bayes from previous class).

Given our labeled corpus $\mathcal{D}_L$, the estimation process consists of counting how
many times each event occurs in the corpus and normalize the counts to
form proper probability distributions. Let us define the following
quantities, called sufficient statistics, that represent the counts of
each event in the corpus:

\begin{align}
\mathbf{Initial \ Counts\!:}\;\;\;\;  &  C_{\mathrm{init}}(c_k) = \sum_{m=1}^M
\Ind (y^m_1 = c_k); \label{eq::initialCounts}\\
%
%\mathbf{Final \ Counts\!:}\;\;\;\;  &  fc(\hv_l,\hv _m) = \sum_{\trex} 
%\Ind (\hs_N = \hv_l \mid \hs_{N-1} = \hv_m); \label{eq::finalCounts}\\
%
\mathbf{Transition \ Counts\!:}\;\;\;\;  &  C_{\mathrm{trans}}(c_k,c_l) =
\sum_{m=1}^M  \sum_{i = 2}^{N}
\Ind (y^m_i = c_k \wedge y^m_{i-1} = c_l); \label{eq::transitionCounts}\\
%
\mathbf{Final \ Counts\!:}\;\;\;\;  &  C_{\mathrm{final}}(c_k) = \sum_{m=1}^M
\Ind (y^m_N = c_k); \label{eq::finalCounts}\\
%
\mathbf{Emission \ Counts\!:}\;\;\;\;  &  
C_{\mathrm{emiss}}(w_j,c_k) = \sum_{m=1}^M
\sum_{i = 1}^{N}
\Ind (x^m_i = w_j \wedge y^m_i = c_k); \label{eq::emissionCounts}
\end{align}
Here $y^m_i$,  the underscript denotes the state index position for a given sequence, and the superscript denotes the sequence index in the dataset, and the same applies for the observations.
Note that $\Ind$ is an indicator function that has the value 1 when the
particular event happens, and zero otherwise. In other words, the previous
equations go through the training corpus and count how
often each event occurs. For example, Eq.~\ref{eq::transitionCounts} counts how many times $c_k$ follows state $c_l$. Therefore, $C_{\mathrm{trans}}(\text{\tt sunny},\text{\tt rainy})$ contains the number of times that a sunny day followed a rainy day.

%
%: e.g. the word ``w'' appears with state
%``s'', or state ``s'' follows another state ``s'', or state ``s''
%begins the sentence.


After computing the counts, one can perform some sanity checks
to make sure the implementation is correct. Summing over all entries
of each count table we should observe the following:

\begin{itemize}
\item \textbf{Initial \ Counts\!:} -- Should sum to the number of
  sentences: $\sum_{k=1}^K C_{\mathrm{init}}(c_k) = M$
%\item \textbf{Final \ Counts\!:} - Should sum to the number of sentences.
\item \textbf{Transition/Final \ Counts\!:} -- Should sum to the number of
  tokens: 
  $\sum_{k,l=1}^K C_{\mathrm{trans}}(c_k,c_l) + \sum_{k=1}^K C_{\mathrm{final}}(c_k) = MN$
%   minus 2 times the number of sentences. Note that there are
%  N-1 edges for each sentence, and the last edge is being accounted by
%  the final transitions. So this leaves us with N-2 edges per sentence,
%  where N is the number of tokens in that sentence.
\item \textbf{Emission \ Counts\!:} -- Should sum to the number of tokens: $\sum_{j=1}^J\sum_{k=1}^K C_{\mathrm{emiss}}(w_j,c_k) = MN$.
\end{itemize}

Using the sufficient statistics (counts) the parameter estimates are: 
\begin{align}
  P_{\mathrm{init}}(c_k | \text{\tt start}) &=  \frac{C_{\mathrm{init}}(c_k)}{\sum_{l=1}^K
    C_{\mathrm{init}}(c_l)}\\
  P_{\mathrm{final}}(\text{\tt stop} | c_l) &=  \frac{C_{\mathrm{final}}(c_l)}{\sum_{k=1}^K
    C_{\mathrm{trans}}(c_k,c_l) + C_{\mathrm{final}}(c_l)}\\
  P_{\mathrm{trans}}(c_k | c_l) &=  \frac{C_{\mathrm{trans}}(c_k,c_l)}{\sum_{p=1}^K
    C_{\mathrm{trans}}(c_p,c_l) + C_{\mathrm{final}}(c_l)}\\
  P_{\mathrm{emiss}}(w_j | c_k) &=  \frac{C_{\mathrm{emiss}}(w_j,c_k)}{\sum_{q=1}^J
    C_{\mathrm{emiss}}(w_q,c_k)}
\end{align}


%\begin{exercise}
%Convince yourself that the sanity checks described above are true.
%Collect the counts from a supervised corpus using method
%\emph{collect\_counts\_from\_corpus} and use the provided function \emph{sanity\_check\_counts} to perform these checks on the counts table. 
%%\begin{python}
%% def sanity_check_counts(self,seq_list):
%%\end{python}
%
%\begin{python}
%>>> import sequences.hmm as hmmc
%>>> hmm = hmmc.HMM(simple.x_dict, simple.y_dict)
%>>> hmm.train_supervised(simple.train)
%>>> hmm.sanity_check_counts(simple.train)
%>>> print "Initial Probabilities:", hmm.initial_probs
%
%>>> print "Transition Probabilities:", hmm.transition_probs
%
%>>> print "Final Probabilities:", print 
%hmm.final_probs
%
%>>> print "Emission Probabilities", print hmm.emission_probs
%
%Initial counts match
%Final counts match
%Transition counts match
%Observations counts match
%\end{python}
%\end{exercise}



\begin{exercise}
The provided function \emph{train\_supervised} from the \emph{hmm.py} file implements the above parameter estimates.
%Implement a function that estimates the maximum likelihood
%estimates for the parameters given the corpus in the class HMM.
%The function header is in the hmm.py file. 
%\begin{python}
%def train_supervised(self,sequence_list):
%\end{python}
Run this function given the simple dataset above and look at the estimated probabilities. Are they correct? You can also check the variables ending in \emph{\_counts} instead of \emph{\_probs} to see the raw counts (for example, typing \emph{hmm.initial\_counts} will show you the raw counts of initial states). How are the counts related to the probabilities?

\begin{python}
>>> import lxmls.sequences.hmm as hmmc
>>> hmm = hmmc.HMM(simple.x_dict, simple.y_dict)
>>> hmm.train_supervised(simple.train)
>>> print "Initial Probabilities:", hmm.initial_probs
[ 0.66666667  0.33333333]
>>> print "Transition Probabilities:", hmm.transition_probs
[[ 0.5    0.   ]
 [ 0.5    0.625]]
>>> print "Final Probabilities:", hmm.final_probs
[ 0.     0.375]
>>> print "Emission Probabilities", hmm.emission_probs
[[ 0.75   0.25 ]
 [ 0.25   0.375]
 [ 0.     0.375]
 [ 0.     0.   ]]
\end{python}
\end{exercise}


%%% Local Variables: 
%%% mode: latex
%%% TeX-master: "../../guide"
%%% End: 
