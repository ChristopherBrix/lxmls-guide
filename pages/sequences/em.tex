Given the training corpus $\mathcal{D}_{U} := \{x^1, \ldots, x^M\}$, training 
seeks model parameters $\theta$ that minimize the negative log-likelihood of the corpus:
\begin{equation}
\label{loglikelihoood}
\mathbf{Negative\;Log\;Likelihood\!:}\;\;\;\; \likelihood(\theta) = \widehat{\mathbb{E}} [-\log P_{\theta}(X)] = -\frac{1}{M}\sum_{m=1}^M \log \left( \sum_{y^m \in \Lambda} P_{\theta}(X=x^m,Y=y^m) \right),
\end{equation}
where $\widehat{\mathbb{E}}[f(X)] := \frac{1}{M}\sum_{m=1}^{M} f(x^m)$ denotes the empirical average of a function $f$ over the training corpus. Because of the hidden variables $y^1,\ldots,y^M$, the likelihood term contains a
sum over all possible hidden structures inside of a logarithm, which
makes this quantity hard to compute.

The most common minimization
algorithm to fit the model parameters in the presence of hidden
variables is the Expectation-Maximization (EM) algorithm. 
The EM procedure can be thought of intuitively in the following way. 
If we observe the hidden variables' values for all sentences in the
corpus, then we could easily compute the maximum likelihood value of
the parameters as described in Section \ref{ml}. 
On the other hand, if we had the model parameters we could label data
using the model, and collect the
sufficient statistics described in Section \ref{ml}.
However, since we are working in an unsupervised setting, we never get to
observe the hidden state sequence. Instead, given the 
training set $\mathcal{D}_{U} = \{x^1 \ldots x^M\}$, we will need to
collect \emph{expected} sufficient statistics (in this case, \emph{expected} counts) 
that
represent the number of times that each hidden variable is
expected to be used with the current parameters setting. These expected sufficient
statistics will then be used during learning as ``fake'' observations of
the hidden variables. Using the node and edge posterior distributions
described in Equations \ref{eq::nodePosterior} and \ref{eq::edgePosterior},
the expected sufficient statistics can 
be computed by the following formulas:

\begin{align}
\mathbf{Initial \ Counts\!:}\;\;\;\;  &  C_{\mathrm{init}}(c_k) = \sum_{m=1}^M
P_{\theta} (Y^m_1 = c_k \,\,|\,\, X^m = x^m); \label{eq::initialCountsPost}\\
%
%\mathbf{Final \ Counts\!:}\;\;\;\;  &  fc(\hv_l,\hv _m) = \sum_{\trex} 
%\Ind (\hs_N = \hv_l \mid \hs_{N-1} = \hv_m); \label{eq::finalCounts}\\
%
\mathbf{Transition \ Counts\!:}\;\;\;\;  &  C_{\mathrm{trans}}(c_k,c_l) =
\sum_{m=1}^M  \sum_{i = 2}^{N}
P_{\theta}(Y^m_i = c_k \wedge Y^m_{i-1} = c_l \,\,|\,\, X^m = x^m); \label{eq::transitionCountsPost}\\
%
\mathbf{Final \ Counts\!:}\;\;\;\;  &  C_{\mathrm{final}}(c_k) = \sum_{m=1}^M
P_{\theta} (Y^m_N = c_k \,\,|\,\, X^m = x^m); \label{eq::finalCountsPost}\\
%
\mathbf{Emission \ Counts\!:}\;\;\;\;  &  
C_{\mathrm{emiss}}(w_j,c_k) = \sum_{m=1}^M
\sum_{i = 1}^{N}
\Ind (x^m_i = w_j) P_{\theta}(Y^m_i = c_k \,\,|\,\, X^m = x^m); \label{eq::emissionCountsPost}
\end{align}

%\begin{align}
%\mathbf{Initial \ Counts\!:}\;\;\;\;  &  ic(\hv_l) = \sum_{d=1}^D
%\gamma_1 (\hv_l); \label{eq::initialCountsPost}\\
%\mathbf{Final \ Counts\!:}\;\;\;\;  &  fc(\hv_N,\hv _{N-1}) = \sum_{d=1}^D  \xi_{N-1} (\hv_l,\hv_m); \label{eq::finalCountsPost}\\
%\mathbf{Transition \ Counts\!:}\;\;\;\;  &  tc(\hv_l,\hv _m) = \sum_{d=1}^D \sum_{i = 1}^{N-1}  \xi_i (\hv_l,\hv_m); \label{eq::transitionCountsPost}\\
%\mathbf{State \ Counts\!:}\;\;\;\;  &  sc(\vv_q,\hv_m) = \sum_{d=1}^D \sum_{i = 1 , \obs_i = \vv_q }^{N}  \gamma_i (\hv_m). \label{eq::stateCountsPost}
%\end{align}

Compare the previous equations with the ones described in Section
\ref{ml} for the same quantities (Eqs.~\ref{eq::initialCounts}--\ref{eq::emissionCounts}). The main difference is that while in
the presence of supervised data you sum the observed events, when you
have no label data you sum the posterior probabilities of each
event. If these probabilities were such that the probability mass was
around single events then both equations will produce the same result.



The EM procedure starts with an initial guess for the parameters
$\theta^0$ at time $t = 0$. The algorithm keeps iterating 
until it converges to a local minima of the negative log likelihood. Each
iteration is divided into two steps:
\begin{itemize} 
 \item The {\bf{E-Step}} (Expectation) 
computes the posteriors for the hidden variables $P_{\theta^t}(Y|X=x^m)$
given the current parameter values $\theta^t$ and the observed variables $X=x^m$ for the $m$-th sentence. 
For an HMM, this can be done through the Forward-Backward algorithm described in the previous sections.
\item The {\bf{M-step}} (Maximization) uses those posteriors $P_{\theta^t}(Y|X=x^m)$ to
``softly fill in'' the values of the hidden variables $Y^m$, and
collects the sufficient statistics: initial counts (Eq: \ref{eq::initialCountsPost}), transition counts (Eq:
\ref{eq::transitionCountsPost}), 
final counts  (Eq: \ref{eq::finalCountsPost}),
and emission counts (Eq: \ref{eq::emissionCountsPost}). These
counts are then used to estimate maximum likelihood parameters $\theta^{t+1}$, as described in
Section \ref{ml}.
\end{itemize}

The EM algorithm is guaranteed to
converge to a local minimum of the log-likelihood $\likelihood(\theta)$, under mild
conditions.  
Note that we are not committing to the best assignment of the hidden variables, but
summing the occurrences of each parameter weighed by the posterior
probability of all possible assignments. 
This modular split into two intuitive and straightforward steps
accounts for the vast popularity of EM.%
\footnote{More formally, EM minimizes an \emph{upper bound} of $\likelihood(\theta)$ via block-coordinate descent. See \citet{Neal1998} for details. Also, even though we are presenting EM in the context of HMMs, this algorithm can be more broadly applied to any probabilistic model with latent variables.} %

% on an upper bound $F(q,\theta)$ using an auxiliary distribution over the latent variables
%$\auxq$:
%\begin{eqnarray}
%\likelihood(\theta)  &=& \XpD \left[-\log \sum_{\hseq}\joint \right]\\
%\label{eq:jensen}&=& \XpD \left[-\log \sum_{\hseq}
%\auxq*\frac{\joint}{\auxq}\right] \le \XpD \left[- \sum_{\hseq} \auxq\log \frac{\joint}{\auxq}\right] \\
%&=& \XpD \left[\sum_{\hseq} \auxq\log \frac{\auxq}{\joint}\right] =  F(q,\theta),
%\end{eqnarray}
%where we have multiplied and divided the $\joint$ by the same quantity
%$\auxq$, and 
%the lower bound comes from applying Jensen Inequality (Equation
%\ref{eq:jensen}). $F(q,\theta)$ is normally referred to as the energy
%function, which comes from the physics field and refers to the energy of a given system that we want to minimize.
%\begin{equation}
%\mathbf{EM\;Upper\;Bound\!:}\;\;\;\;\;\likelihood(\theta) \le F(q,\theta) =
%\XpD \left[\sum_{\hseq} \auxq\log \frac{\auxq}{\joint}\right].
%\end{equation}
%The alternating E and M steps at iteration $t+1$ can be seen as minimizing the energy function first 
%with respect to $\auxq$ and then with respect to $\theta$:
%\begin{eqnarray}
%\hspace{-5mm}{\mathbf E\!:}&& q^{t+1}(\hseq\mid\sent) =
%\argmin_{\auxq} F(q,\theta^t)
%  = \argmin_{\auxq} \KL(q(\hseq\mid\sent)\,||\,p_{\theta^t}(\hseq\mid\sent)) = p_{\theta^t}(\hseq\mid\sent);
% \label{eq:e-step} \\
%\hspace{-5mm}{\mathbf M\!:}&& \theta^{t+1} = \argmin_\theta
%F(q^{t+1},\theta) = \argmax_\theta \XpD\!\left[\sum_{\hseq}
%q^{t+1}(\hseq\mid\sent)\log p_\theta(\sent,\hseq)\right];
%\label{eq:m-step}
%\end{eqnarray}
%where $\KL(q||p) = \Xp_q[\log \frac{q(\cdot)}{p(\cdot)}]$ is the
%Kullback-Leibler divergence. The KL term in the E-Step results from 
%dropping all terms from the energy function that are constant for a
%set $\theta$, in this case the likelihood of the observation sequence
%$\marginal$:
%
%\begin{align}
%\sum_{\hseq} \auxq\log \frac{\auxq}{\joint} &= \sum_{\hseq} \auxq\log
%\auxq - \sum_{\hseq} \auxq\log \joint  \\ 
%&= \sum_{\hseq} \auxq\log \auxq - \sum_{\hseq} \auxq\log \marginal
%\posterior \\
%&= \sum_{\hseq} \auxq\log \frac{\auxq}{\posterior} - \log \marginal \\
%&= \KL(\auxq||\posterior) - \log \marginal.
%\end{align}

Algorithm ~\ref{alg::em} presents the pseudo code for the EM
algorithm. 
%Note that this algorithm is agnostic of a particular model,
%it only requires the model to implement a common interface.

\begin{algorithm}
\begin{algorithmic}[1]
  \STATE {\bfseries input:} dataset $\mathcal{D}_{U}$, initial model $\theta$
    \FOR{$t = 1$ {\bfseries to} $T$}
    	  \STATE
    	  \STATE \textbf{E-Step:}
      \STATE Clear counts: $C_{\mathrm{init}}(.) = C_{\mathrm{trans}}(.,.) = C_{\mathrm{final}}(.) = C_{\mathrm{emiss}}(.,.) = 0$
      \FOR{$x^m \in \mathcal{D}_{U}$}
      	\STATE Compute posterior expectations $P_{\theta}(Y_i|X=x^m)$ and $P_{\theta}(Y_i,Y_{i+1}|X=x^m)$ using the current model $\theta$
%        \STATE posteriors,likelihood =model.compute\_posteriors($seq$)
%        \STATE model.update\_counts($seq$,posteriors)
      \STATE Update counts as shown in Eqs.~\ref{eq::initialCountsPost}--\ref{eq::emissionCountsPost}.
      \ENDFOR
    	  \STATE
      \STATE \textbf{M-Step:}
         \STATE Update the model parameters $\theta$ based on the counts.
    \ENDFOR
\end{algorithmic}    
\caption[EM algorithm]{\label{alg::em}  EM algorithm.} 
\end{algorithm}


One important thing to note in Algorithm \ref{alg::em} is that for the
HMM model we already have all the model pieces we require. In fact
the only method we don't have yet implemented from previous classes is
the method to update the counts given the posteriors. 

\begin{exercise}

Implement the method to update the counts given the state and transition posteriors.
\begin{python}
 def update_counts(self, sequence, state_posteriors, transition_posteriors):
\end{python}

%Use the method you defined previously to check the count tables to
%check if this method is correct. Use a corpus with only one sentence
%to make the test simpler.

%\begin{python}
%In []: run readers/pos_corpus.py
%In []: posc = PostagCorpus("en",max_sent_len=15,train_sents=1,dev_sents=0,test_sents=0)
%In []: run sequences/hmm.py
%In []: hmm = HMM(posc)
%In []: hmm.train_supervised(posc.train,smoothing=0.1)
%In []: hmm.clear_counts()
%In []: posteriors,likelihood = hmm.get_posteriors(posc.train.seq_list[0])
%In []: hmm.update_counts(posc.train.seq_list[0],posteriors)
%In []: hmm.sanity_check_counts(posc.train)
%\end{python}

%If you pass this test, then you have all the pieces to implement the
%EM algorithm. Look at the code for EM algorithm in file
%\emph{sequences/em.py} and check it for yourself. 

You now have all the pieces to implement the
EM algorithm. Look at the code for EM algorithm in file
\emph{sequences/hmm.py} and check it for yourself. 

\begin{python}
    def train_EM(self, dataset, smoothing=0, num_epochs=10, evaluate=True):
        self.initialize_random()

        if evaluate:
            acc = self.evaluate_EM(dataset)
            print "Initial accuracy: %f"%(acc)
            
        for t in xrange(1, num_epochs):
            #E-Step
            total_log_likelihood = 0.0
            self.clear_counts(smoothing)
            for sequence in dataset.seq_list:
                # Compute scores given the observation sequence.
                initial_scores, transition_scores, final_scores, emission_scores = \
                    self.compute_scores(sequence)
                
                state_posteriors, transition_posteriors, log_likelihood = \
                    self.compute_posteriors(initial_scores,
                                            transition_scores,
                                            final_scores,
                                            emission_scores)

                self.update_counts(sequence, state_posteriors, transition_posteriors)
                total_log_likelihood += log_likelihood
            print "Iter: %i Log Likelihood: %f"%(t, total_log_likelihood)
            #M-Step
            self.compute_parameters()
            if evaluate:
                 ### Evaluate accuracy at this iteration
                acc = self.evaluate_EM(dataset)
                print "Iter: %i Accuracy: %f"%(t,acc)
\end{python}

\end{exercise}

%%% Local Variables: 
%%% mode: latex
%%% TeX-master: "../../guide"
%%% End: 
